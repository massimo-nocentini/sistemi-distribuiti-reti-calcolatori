\documentclass[11pt]{article}
\usepackage{fullpage}
\usepackage{sdrcform}

\pagesettings

%%%%%%%%%%%%%%%%%%%%%%%%%
% Intestazione
% Campi da non modificare
\newcommand{\degreename}{Informatica}
\newcommand{\coursename}{SDRC}
\newcommand{\professorname}{Pierluigi Crescenzi}
\newcommand{\professoremail}{pierluigi.crescenzi@unifi.it}
\newcommand{\issuedate}{4 Ottobre 2013}
% Campi da riempire
% Il titolo  (quello della dispensa)
\newcommand{\notetitle}{Algoritmo veloce per la determinazione
  dell'insieme dominante}
% L'autore (gli autori)  (sono) quello (quelli) della dispensa
\newcommand{\authornames}{Alessandro Venturi}
% Il lettore  (chi sta compilando il modulo)
\newcommand{\refereename}{Massimo Nocentini}
\newcommand{\refereeemail}{massimo.nocentini@gmail.com}
% Data di consegna del modulo
\newcommand{\preparedate}{24 Dicembre 2013}

%%%%%%%%%%%%%%%%%%%%%%%%%
% Giudizi (mettere il voto nelle parentesi graffe)
% Valutazione globale
\newcommand{\recommendation}{2}
% 1=Ottimo/2=Distinto/3=Buono/4=Discreto/5=Sufficiente
% Come giudicheresti le tue conoscenze nel settore?
\newcommand{\competence}{2} 
% 1=Molto esperto/2=Esperto/3=Interessato/4=Competente
% Quanto sei sicuro del tuo giudizio?
\newcommand{\confidence}{1}
% 1=Molto confidente/2=Confidente/3=Abbastanza confidente
% Quanto sforzo ha richiesto leggere la dispensa?
\newcommand{\effort}{2}
% 1=Grande/2=Ragionevole/3=Poco/4=Quasi nullo
% Quanto bene hai capito la dispensa?
\newcommand{\comprehension}{2}
% 1=Perfettamente/2=Maggioranza/3=Idee/4=Abbastanza confuso/5=Molto confuso
% Quanti dettagli hai verificato?
\newcommand{\details}{1} 
% 1=Tutti/2=Maggior parte/3=Abbastanza/4=Pochi
% Quanto le note integrano quanto detto a lezione?
\newcommand{\originality}{1}
% 1=Molto/2=Abbastanza/3=Poco/4=Per nulla
% Come giudichi le dimostrazioni incluse nelle note?
\newcommand{\proofs}{2} 
% 1=Belle/2=Piacevoli/3=Allo scopo/4=Brutte
% Come giudichi i risultati inclusi nelle note?
\newcommand{\correctness}{2}
% 1=Corretti/2= Quasi sicuramente corretti/3=Probabilmente corretti/4= Probabilmente non corretti  
% 5=Dimostrazioni sbagliate/6=Risultati sbagliati/7=Non so dirlo
% Quale livello di conoscenze e' necessario per leggere le note?
\newcommand{\accessibility}{4}
% 1=Esperto/2=Specialista/3=Teorico/4=Informatico
% Quanto bene sono scritte le dispense?
\newcommand{\presentation}{1}
% 1=Quasi senza errori/2=Buona/3=Adeguata/4=Superficiale/5=Incomprensibile
% Quanto lunghe sono le dispense?
\newcommand{\density}{2} 
% 1=Troppo/2=Adeguato/3=Poco/4=Troppo poco
% Cosa pensi dei dettagli che mancano?
\newcommand{\missing}{6}
% 1=Probabilmente sbagliati/2=Spesso incomprensibili/3=Grande sforzo
% 4=Sforzo/5=Piccolo sforzo/6=Nessuno sforzo/7=Troppi dettagli
% Cosa pensi della qualita' dell'italiano?
\newcommand{\italian}{2}
% 1=Eccellente/2=Buono/3=Adeguato/4=Sotto lo standard
% 5=Non adeguato/6=Pessimo

%%%%%%%%%%%%%%%%%%%%%%%%%
% Commenti (da inserire obbligatoriamente)
% Inserisci i tuoi commenti, ovvero una valutazione qualitativa
% della dispensa. Tali commenti devono anche spiegare le
% valutazioni quantitative date in precedenza
\begin {document}
\theform
\begin{formalreport}
  Ci sono due piccoli typo a pagina 2: $(i)$ $u\in w(v)$ dovrebbe
  essere $u\in W(v)$ ed $(ii)$ equazione 5, in una sommatoria dovrebbe
  essere rimosso l'estremo superiore $k$ in quanto si quantifica su
  elementi dell'insieme $W (v)$.

  Non ho dato il massimo del punteggio sulle dimostrazioni in quanto
  la dimostrazione del numero di passi dell'algoritmo a pagina 4
  contiene errori (si \`e sempre considerato l'evento
  complementare a quello a cui siamo interessati), riporto sotto la
  versione secondo me corretta:
  \begin{itemize}
  \item \emph{Pr} (``$u$ sia bianco al passo successivo'') $\leq
    \frac{8}{9}$
  \item \emph{Pr} (``$u$ sia bianco al $\tau$-esimo passo
    successivo'') $\leq \left(\frac{8}{9}\right)^{\tau}$
  \item $\tau = \log_{\frac{9}{8}} 2n \rightarrow $ \emph{Pr} (``$u$
    sia bianco al $\tau$-esimo passo successivo'') $\leq
    \frac{1}{2n}$
  \item applicando union bound: $\sum_{u\in V}$ \emph{Pr} (``$u$ sia
    bianco al $\tau$-esimo passo successivo'') $\leq \frac{1}{2}$
  \item $\tau =k \log_{\frac{9}{8}} 2n \rightarrow \sum_{u\in V}$
    \emph{Pr} (``$u$ sia bianco al $\tau$-esimo passo successivo'')
    $\leq \left(\frac{1}{2}\right)^k$, rendendola arbitrariamente
    piccola in base alla scelta di $k$

  \end{itemize}
\end{formalreport}

\end {document}
