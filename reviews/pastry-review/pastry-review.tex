\documentclass[11pt]{article}
\usepackage{fullpage}
\usepackage{sdrcform1}
\usepackage[utf8x]{inputenc} 

\pagesettings

%%%%%%%%%%%%%%%%%%%%%%%%%
% Intestazione
% Campi da non modificare
\newcommand{\degreename}{(Ingegneria) Informatica}
\newcommand{\coursename}{SDRC}
\newcommand{\professorname}{Pierluigi Crescenzi}
\newcommand{\professoremail}{pierluigi.crescenzi@unifi.it}
\newcommand{\issuedate}{4 Ottobre 2013}
% Campi da riempire
% Il titolo  (quello della dispensa)
\newcommand{\notetitle}{Pastry}
% L'autore (gli autori)  (sono) quello (quelli) della dispensa
\newcommand{\authornames}{Carli, Cimato, Salvi}
% Il lettore  (chi sta compilando il modulo)
\newcommand{\refereename}{Pierucci, Nocentini, Bruni}
\newcommand{\refereeemail}{}
% Data di consegna del modulo
\newcommand{\preparedate}{31 Dicembre 2013}

%%%%%%%%%%%%%%%%%%%%%%%%%
% Giudizi (mettere il voto nelle parentesi graffe)
% Valutazione globale
\newcommand{\recommendation}{3}
% 1=Ottimo/2=Distinto/3=Buono/4=Discreto/5=Sufficiente
% Come giudicheresti le tue conoscenze nel settore?
\newcommand{\competence}{3} 
% 1=Molto esperto/2=Esperto/3=Interessato/4=Competente
% Quanto sei sicuro del tuo giudizio?
\newcommand{\confidence}{2}
% 1=Molto confidente/2=Confidente/3=Abbastanza confidente
% Quanto sforzo ha richiesto leggere la dispensa?
\newcommand{\effort}{2}
% 1=Grande/2=Ragionevole/3=Poco/4=Quasi nullo
% Quanto bene hai capito la dispensa?
\newcommand{\comprehension}{2}
% 1=Perfettamente/2=Maggioranza/3=Idee/4=Abbastanza confuso/5=Molto confuso
% Quanti dettagli hai verificato?
\newcommand{\details}{2} 
% 1=Tutti/2=Maggior parte/3=Abbastanza/4=Pochi
% Quanto le note integrano quanto detto a lezione?
\newcommand{\originality}{2}
% 1=Molto/2=Abbastanza/3=Poco/4=Per nulla
% Come giudichi le dimostrazioni incluse nelle note?
\newcommand{\proofs}{3} 
% 1=Belle/2=Piacevoli/3=Allo scopo/4=Brutte
% Come giudichi i risultati inclusi nelle note?
\newcommand{\correctness}{2}
% 1=Corretti/2= Quasi sicuramente corretti/3=Probabilmente corretti/4= Probabilmente non corretti  
% 5=Dimostrazioni sbagliate/6=Risultati sbagliati/7=Non so dirlo
% Quale livello di conoscenze e' necessario per leggere le note?
\newcommand{\accessibility}{4}
% 1=Esperto/2=Specialista/3=Teorico/4=Informatico
% Quanto bene sono scritte le dispense?
\newcommand{\presentation}{3}
% 1=Quasi senza errori/2=Buona/3=Adeguata/4=Superficiale/5=Incomprensibile
% Quanto lunghe sono le dispense?
\newcommand{\density}{2} 
% 1=Troppo/2=Adeguato/3=Poco/4=Troppo poco
% Cosa pensi dei dettagli che mancano?
\newcommand{\missing}{5}
% 1=Probabilmente sbagliati/2=Spesso incomprensibili/3=Grande sforzo
% 4=Sforzo/5=Piccolo sforzo/6=Nessuno sforzo/7=Troppi dettagli
% Cosa pensi della qualita' dell'italiano?
\newcommand{\italian}{2}
% 1=Eccellente/2=Buono/3=Adeguato/4=Sotto lo standard
% 5=Non adeguato/6=Pessimo

%%%%%%%%%%%%%%%%%%%%%%%%%
% Commenti (da inserire obbligatoriamente)
% Inserisci i tuoi commenti, ovvero una valutazione qualitativa
% della dispensa. Tali commenti devono anche spiegare le
% valutazioni quantitative date in precedenza
\begin {document}
\theform
\begin{formalreport}
  \textbf{Commenti}\\
  Il lavoro del gruppo lo riteniamo sufficiente in quanto non abbiamo
  trovato spunti individuali che lo differenziano dall'articolo
  originale: in molti versi i paragrafi riportati sono nello stesso
  ordine, con le stesse frasi e stessa cadenza. 

  L'italiano usato ci risulta abbastanza fluido, sono presenti alcuni
  typo ed avremo preferito una maggior continuit\`a tra la divisione
  in sezioni.

  Apprezzabile il tentativo di dar risalto con due teoremi a due idee
  fondamentali e l'aggiunta di Figura 2 e Figura 5 per aiutare
  l'esposizione.

  \textbf{Dispensa}\\
  La dispensa \`e stata strutturata in modo molto simile all'articolo
  originale. Dopo un'introduzione al contesto in cui il lavoro si
  colloca, segue una sezione in cui viene descritta l'overlay network
  \emph{Pastry}, dando particolare risalto alle informazioni che ogni
  nodo mantiene. La descrizione della tabella di routing ci \`e
  sembrata poco chiara in quanto, con fatica, si distinguono i
  soggetti nelle frasi che spiegano il significato delle righe della
  tabella. Nelle definizioni dei due insiemi \emph{neighborhood} e
  \emph{leaf} si utilizzano rispettivamente i concetti di
  \emph{metrica di prossimit\`a} e di \emph{vicino} di cui non si
  danno ulteriori informazioni, anche nel seguito verranno lasciati
  abbastanza vaghi.

  \`E apprezzabile il tentativo di dare risalto al Teorema 1.1
  rispetto all'articolo, dove l'idea e la sua prova vengono date in un
  paragrafo senza particolare enfasi. Nonostante questo, nella prova
  non abbiamo capito il risultato nell'ottica dell'alta probabilit\`a
  come definita a lezione.

  La dispensa non riporta una sezione nella quale si mette in evidenza
  una possibile interfaccia a scopo implementativo: riteniamo che
  sarebbe stata utile per capire quali servizi lo strato di
  indirezione introdotto da Pastry offre alle applicazioni che
  desiderano sfruttare le sue caratteristiche, e quali eventi possono
  essere gestiti in seguito al routing dei messaggi.

  Nella Sezione 4 riguardante l'aggiunta di nodi, vi \`e discrepanza
  rispetto all'articolo in quanto si riporta ``$x$, che avr\`a in
  generale un nodeId distante da quello di $a$, invia quindi uno
  speciale messaggio di join ad $a$ con chiave $x$'', mentre
  nell'originale $x$ invia una richiesta di join ad $a$,
  successivamente \`e $a$ a fare il routing di un pacchetto contente
  $x$ come chiave. Per quanto riguarda la costruzione della routing
  table del nuovo nodo $x$ non viene giustificata la strategia
  riportata, a differenza dell'articolo in cui, in pi\`u, viene fatta
  una osservazione sul carattere ``ottimistico'' di tale strategia.

  Nella Dimostrazione 1.2 non ci \`e chiaro perch\`e la costruzione
  della tabella $R$, usando le informazioni delle tabelle dei nodi
  attraversati nel cammino $a \rightarrow \ldots \rightarrow z$,
  preserva la propriet\`a di \emph{localit\`a}: questa dimostrazione
  non \`e del tutto formale neppure nell'articolo di studio.

  Nella Sezione 6.2 il risultato riportato non \`e coerente in quanto,
  nello stesso paragrafo, si arriva a conclusioni opposte (forse una
  svista nel secondo punto a met\`a pagina).

  \textbf{Presentazione}\\
  La presentazione ci \`e sembrata spedita anche se non del tutto
  chiara, in particolare le tabelle riportate ci sono sembrate
  complesse per una prima esposizione dell'argomento.
 
\end{formalreport}

\end {document}
