\documentclass{article}
\usepackage{sdrc}
\usepackage[ruled,vlined]{algorithm2e}
\lecture{1}

\def\title{Massimo Nocentini}

\begin{document}

\maketitle

\section{Analisi della complessit\`a dell'algoritmo \emph{MIS}}
Sia $(\Omega, \mathcal{A}, \mathbf{P})$ uno spazio di probabilit\`a.
Nel seguito assumeremo di lavorare su un grafo $G = (V, E)$ dove, come
di consueto, $V$ rappresenta l'insieme dei vertici ed $E$ rappresenta 
l'insieme degli archi.

\begin{defn}{Evento con alta probabilit\`a}
Un evento $E \in \mathcal{A}$ si dice che si verifica \emph{
con alta probabilit\`a} se 
\begin{displaymath}
    \mathbf{P}(E) \geq 1 - {{1} \over {n^\alpha}} 
    %\quad n, \alpha > 0
\end{displaymath}
dove $n = |V|$ e $\alpha > 0$.
\end{defn}

Possiamo vedere l'esecuzione dell'algoritmo come una successione di fasi
$\lbrace t_i \rbrace_{i \in \N}$, ognuna con durata $\epsilon = 
{{1} \over {4\e^{4}}}$

% dare la definizione di superfase.
Quello che vogliamo far vedere \`e che alla fine di ogni super-fase il 
numero di nodi da considerare nella super-fase successiva viene 
dimezzato con alta probabilit\`a. Prima di dare una scrittura formale di questo 
teorema introduciamo i seguenti oggetti:
\begin{itemize}
    \item $U_t = \lbrace u \in V : u \text{ non ha deciso all'inizio della
        fase } t \rbrace$;
    \item $G_t = (U_t, E_t)$ \`e il grafo indotto dall'insieme di nodi $U_t$;
    \item $M_i(t) = \lbrace u \in U_t : \delta_{G_t}(u) \geq 
        \frac{n}{2^i} \rbrace$, dove $\delta_{G_t}(u)$ rappresenta il grado
        del nodo $u$ nel grafo $G_t$.
\end{itemize}

\begin{lemma}
    \label{lemma:epsilon-i}
    Sia $t_i$ una fase dell'algoritmo e $\mathcal{E}_i \in \mathcal{A}$ 
    tale che $\mathcal{E}_i = \lbrace M_i(t_i) = \emptyset\rbrace$. \\
    Allora $\mathcal{E}_i$ si verifica con alta probabilit\`a.
\end{lemma}


    L'idea della prova \`e quella di trovare una stima della probabilit\`a 
    del verificarsi dell'evento $\mathcal{E}_{i+1}$ assumendo che l'evento 
    $\mathcal{E}_{i}$ si sia verificato: per far questo faremo vedere che
    un eventuale vertice $v \in M_{i+1}(t_{i+1})$ esiste con bassa
    probabilit\`a e, se questo \`e il caso, il vertice $v$ ``sparisce''
    tra le fasi $t_i, t_{i+1}$ con alta probabilit\`a.

\begin{proof}
    Seguendo l'idea precedente studiamo il primo evento $\mathcal{E}_0$. Questo 
    \`e il caso della prima fase dove nessun nodo $u \in V$ ha un grado 
    $\delta_G(u) \geq n$: il caso che pu\`o dar fastidio \`e per
    $\delta_G(u) = n$ ma lo scartiamo in quanto non ammettiamo self loop. 
    Quindi $M_0(t_0) = \emptyset$ che equivale a dire
    che l'evento $\mathcal{E}_0$ si \`e verificato, ovvero 
    \begin{displaymath}
        \mathbf{P} (\mathcal{E}_0) = 1 \geq 1 - \frac{1}{n^\alpha} \quad
        \forall \alpha > 0
    \end{displaymath}    
     come richiesto.

    Studiamo adesso per un generico evento $\mathcal{E}_{i+1}$ assumendo il
    verificarsi dell'evento $\mathcal{E}_{i}$ e facciamo vedere che
    \begin{displaymath}
        \mathbf{P}(\mathcal{E}_{i+1} | \mathcal{E}_{i}) \geq 1 - \frac{1}{n^2}
    \end{displaymath}
    quindi scegliendo $\alpha = 2$ avremo che l'evento 
    $\mathcal{E}_{i+1} | \mathcal{E}_{i}$ si verifica con alta probabilit\`a.

    Supponiamo esista un vertice $v$ tale che $v \in M_{i+1}(t_{i+1})$. Per
    la definizione di $M_{i+1}(t_{i+1})$ allora $\delta_{G_{t_{i+1}}}(v) 
    \geq \frac{n}{2^{i+1}}$. Osserviamo adesso che durante l'avanzamento delle
    fasi il grado di $v$ non pu\`o aumentare in quanto ad ogni fase si 
    rimuovono vertici e non ne vengono aggiunti, quindi $\forall t: 
    t < t_{i+1} \impliessymbol \delta_{G_{t}}(v) \geq \frac{n}{2^{i+1}}$.

    Per ipotesi $\mathcal{E}_i$ si \`e verificato, quindi $\forall t: 
    t \geq t_i \impliessymbol \delta_{G_{t}}(w) < \frac{n}{2^i}$.
    Dato che questo vale per ogni vertice $w \in V_t$ allora vale anche 
    per il vertice di grado massimo, di conseguenza 
    $\Delta(G_{t}) < \frac{n}{2^i}$. Manipolando otteniamo
    $\frac{\Delta(G_{t})}{2} < \frac{n}{2^{i+1}}$ e per transitivit\`a 
    \begin{displaymath}
        \delta_{G_t}(v) \geq \frac{\Delta(G_{t})}{2} \quad \text{dove } 
        t \in [t_i,t_{i+1}]
    \end{displaymath}

    Per la condizione vista nella lezione precedente, vale 
    $\forall t \in [t_i,t_{i+1}]: \mathbf{P}(v \not \in IS) \geq \epsilon$. 
    L'evento $v \not \in IS$ significa che $v$ non appartiene all'insieme
    indipendente e quindi significa che ha deciso, di conseguenza 
    $v \not \in M_{i+1}(t_{i+1})$ \`e un evento equivalente. Possiamo
    riscrivere ed usare il complemento, indicando con $k$ la lunghezza
    di una generica fase:
    \begin{displaymath}
    \begin{array} {lcl} 
        \mathbf{P}(v \not \in M_{i+1}(t_{i+1})) & \geq & \epsilon \\
        \mathbf{P}(v \in M_{i+1}(t_i + j)) & < & 1-\epsilon \quad 
            \text{dove } j\in[1,k]
    \end{array}
    \end{displaymath}
    Osserviamo che la probabilit\`a dell'evento ``$v$ soppravvive a tutti
    gli istanti nell'intervallo $[1,k]$'' \`e maggiorata da:
    \begin{displaymath}
        \mathbf{P}(v \in M_{i+1}(t_{i+1})) \leq (1 - \epsilon)^k 
            \quad \text{dove } k = \frac{\log{n}^3}{\epsilon}
    \end{displaymath}
    Usando il limite notevole $\lim_{x \rightarrow \infty}{(1 + 
    \frac{a}{x})^{bx}} = e^{ab}$ (nel nostro contesto $\epsilon \rightarrow 0
    , a = -1, b = \log{n^3}$)
    segue :
    \begin{displaymath}
        \mathbf{P}(v \in M_{i+1}(t_{i+1})) \leq (1 - \epsilon)^
        \frac{\log{n}^3}{\epsilon} = e^{-\log{n}^3} = \frac{1}{n^3}
    \end{displaymath}
    Osserviamo adesso che la maggiorazione precedente riguarda \emph{un}
    generico vertice $v$, studiando il caso peggiore nel quale tutti i 
    nodi sono sopravvissuti si ottiene:
    \begin{displaymath}
        \mathbf{P}(\Exi{v} \in M_{i+1}(t_{i+1})) \leq 
        \sum_{v \in V_{t_{i+1}}}{\mathbf{P}(v \in M_{i+1}(t_{i+1}))} \leq
        \sum_{v \in V_{t_{i+1}}}{\frac{1}{n^3}} \leq
        \frac{1}{n^2}
    \end{displaymath}
    Per complemento, la probabilit\`a dell'evento $\mathcal{E}_{i+1}|
    \mathcal{E}_i$ \`e almeno $1 - \frac{1}{n^2}$ come richiesto.
\end{proof}

Studiamo adesso la terminazione dell'algoritmo stimando il tempo necessario
affinch\`e tutti i nodi abbiano deciso.

\begin{theorem}
    Dopo $O(\log{n})$ fasi tutti i nodi hanno deciso con alta probabilit\`a.
\end{theorem}

\begin{proof}
    Definiamo i seguenti oggetti che verranno utilizzati nella prova:
    \begin{itemize}
        \item $l = \ceil{\log{n}}$;
        \item $\mathcal{E} = \bigcap_{i \in 
            \lbrace 0, \ldots, l \rbrace}{\mathcal{E}_i}$. Osserviamo che 
            %$\mathcal{E} \in \mathcal{A}$
            , essendo $\mathcal{A}$  chiusa 
            rispetto sia all'unione che all'intersezione, $\mathcal{E}$ 
            \`e un evento in quanto intersezione di eventi $\mathcal{E}_i 
            \in \mathcal{A}$;
        \item $\mathcal{P} \in \mathcal{A}$ tale che $\mathcal{P} = 
            \lbrace \text{al tempo $t_l$ tutti i nodi hanno deciso} \rbrace$.
    \end{itemize}

    L'obiettivo della prova \`e quello di far vedere che l'evento 
    $\mathcal{P}$ si verifica con alta probabilit\`a e successivamente
    che \`e necessario $O(\log{n})$ tempo per la terminazione. 

    Osserviamo che $\mathcal{E}\subseteq\mathcal{P}$ (in parole $\mathcal{E}$ 
    implica $\mathcal{P}$) in quanto l'intersezione tra due eventi 
    $\mathcal{E}_i \cap \mathcal{E}_{i+1} \subseteq \mathcal{E}_{i+1}$,
    e quindi iterando per l'intera intersezione si ha $\mathcal{E} 
    \subseteq \mathcal{E}_l$. Ma se si verifica $\mathcal{E}_l$ allora
    $M_l(t_l) = \emptyset$, ovvero tutti i nodi hanno deciso entro la fase
    $t_l$, che \`e la definizione dell'evento $\mathcal{P}$.

    Ragioniamo per 
    complemento e studiamo l'evento $\overline{\mathcal{E}}$: per le 
    leggi di De Morgan vale $\overline{\mathcal{E}} = (\overline{\mathcal{E}}_1
    \cap \mathcal{E}_0) \cup (\overline{\mathcal{E}}_2
    \cap \mathcal{E}_1 \cap \mathcal{E}_0) \cup \ldots \cup 
    (\overline{\mathcal{E}}_l \cap \mathcal{E}_{l-1} \cap \ldots
    \cap \mathcal{E}_0)$. Per lo stesso ragionamento fatto sull'intersezione di
    eventi $\mathcal{E}_i$ possiamo scrivere $\overline{\mathcal{E}} \subseteq
    (\overline{\mathcal{E}}_1
    \cap \mathcal{E}_0) \cup (\overline{\mathcal{E}}_2
    \cap \mathcal{E}_1 ) \cup \ldots \cup 
    (\overline{\mathcal{E}}_l \cap \mathcal{E}_{l-1})$. Applichiamo adesso 
    la monotonia di $\mathbf{P}$: $\mathbf{P}(\overline{\mathcal{E}}) \leq
    \mathbf{P}(\overline{\mathcal{E}}_1
    \cap \mathcal{E}_0) + \mathbf{P}(\overline{\mathcal{E}}_2
    \cap \mathcal{E}_1 ) + \ldots + 
    \mathbf{P}(\overline{\mathcal{E}}_l \cap \mathcal{E}_{l-1})$. Per la 
    definizione di evento condizionato, se $A,B$ sono eventi allora $
    \mathbf{P}(A|B) = \frac{\mathbf{P}(A\cap B)}{\mathbf{P}(B)}$, quindi vale 
    la maggiorazione ${\mathbf{P}(A\cap B)} \leq 
    \frac{\mathbf{P}(A\cap B)}{\mathbf{P}(B)}$ in quanto $\mathbf{P}(B) \leq 1$
    , allora: 
    $\mathbf{P}(\overline{\mathcal{E}}) \leq
    \mathbf{P}(\overline{\mathcal{E}}_1
    \cap \mathcal{E}_0) + \mathbf{P}(\overline{\mathcal{E}}_2
    \cap \mathcal{E}_1 ) + \ldots + 
    \mathbf{P}(\overline{\mathcal{E}}_l \cap \mathcal{E}_{l-1}) \leq
    \mathbf{P}(\overline{\mathcal{E}}_1
    | \mathcal{E}_0) + \mathbf{P}(\overline{\mathcal{E}}_2
    | \mathcal{E}_1 ) + \ldots + 
    \mathbf{P}(\overline{\mathcal{E}}_l | \mathcal{E}_{l-1})$. Per il 
    lemma precedente vale $\mathbf{P}(\mathcal{E}_{k+1}|
    \mathcal{E}_{k}) < \frac{1}{n^2}$, quindi possiamo maggiorare ancora:
    $\mathbf{P}(\overline{\mathcal{E}}_1
    | \mathcal{E}_0) + \mathbf{P}(\overline{\mathcal{E}}_2
    | \mathcal{E}_1 ) + \ldots + 
    \mathbf{P}(\overline{\mathcal{E}}_l | \mathcal{E}_{l-1}) \leq 
    \frac{1}{n^2} + \ldots + \frac{1}{n^2} \leq \frac{l}{n^2} \leq 
    \frac{\log{n} + 1}{n^2} \leq \frac{1}{n}$. Per complemento abbiamo 
    $\mathbf{P}(\mathcal{E}) \geq 1 - \frac{1}{n}$ quindi, scegliendo $\alpha
    = 1$, l'evento $\mathcal{E}$ si verifica con alta probabilit\`a. Ma
    dato che $\mathcal{E} \impliessymbol \mathcal{P}$, anche l'evento $
    \mathcal{P}$ si verifica con alta probabilit\`a. Pertanto dopo $\log{n}$
    fasi, i nodi hanno deciso, ogni fase ha durata $\frac{3\log{n}}{\epsilon}$
    ed il tempo totale necessario alla terminazione \`e $O(\log{n}^2) = 
    O(\log{n})$ come richiesto.
\end{proof}

\section{Un secondo algoritmo}
Vediamo un secondo algoritmo che permette di costruire un MIS, anche questo
usa un approccio randomizzato. Questa nuovo implementazione assume che ogni
vertice sia a conoscenza dei propri vicini.
\begin{algorithm}
 \SetAlgoLined
 \KwData{Un grafo non orientato $G = (V, E)$}
 \KwResult{insieme di vertici $V' \subseteq V$ tale che $V'$ \`e un MIS}
    \For{$v \in V$}{
        $v$ sceglie $r_v \in (0,1)$\;
        $W_v \leftarrow \emptyset$\;
        \For{$w \in N(v)$}{
            $v$ riceve $r_w$ e aggiunge $r_w$ a $W_v$
        }
        \If{$r_v \leq \min_{r_w \in W_v}\lbrace r_w \rbrace$}{
            $v$ comunica ad ogni $w \in N(v)$ di cancellarsi dal grafo
        }
        \If{$v$ riceve un messaggio di cancellarsi dal grafo da $w \in N(v)$}{
            $v$ comunica ad ogni $s \in N(v)\setminus\lbrace w\rbrace$ 
            di non considerarlo come vicino nella fase successiva
        }
    }
    il grafo iniziale pu\`o dividersi in sottografi, potenzialmente
    non connessi, quindi si invoca ricorsivamente su ognuno di essi
 \caption{Un secondo algoritmo randomizzato per la costruzione di un MIS}
\end{algorithm}

Quello che si pu\`o dimostrare \`e che ad ogni fase il numero medio di archi
si dimezza: per farlo \`e necessario non contare due volte un arco che
viene cancellato a causa di due messaggi diversi. Questa situazione si
verifica nella seguente configurazione che rappresenta uno spezzone della
computazione precedente:
\begin{itemize}
    \item $(x,y),(y,z),(z,w) \in E$;
    \item $r_x < r_y \wedge r_w < r_z$.
\end{itemize}
In questo caso i nodi $x,w$ inviano un messaggio di cancellazione ai nodi $
y,z$ rispettivamente e quindi $y,z$ si ``avvisano'' a vicenda di non
considerarsi vertici vicini nella prossima fase, usando due volte l'arco
$(y,z)$.

\section{Applicazioni}

\subsection{Matching Problem}
\`E possibile risolvere il problema del matching usando l'algoritmo
randomizzato per MIS. Dato un grafo $G = (V,E)$ di cui vogliamo trovare 
un matching massimale possiamo costruire un grafo $G'$ che ha per vertici
gli archi di $G$ e per archi coppie $(e_i, e_j)$ tali che $e_i, e_j \in E
\wedge e_i \cap e_j \not = \emptyset$. In parole, nel nuovo grafo si collegano
due archi se questi sono adiacenti nel grafo originale.

Possiamo applicare l'algoritmo per trovare un insieme indipendente nel nuovo
grafo $G'$ che equivale a trovare un insieme di archi che non sono adiacenti nel
grafo originale, ovvero abbiamo risolto il problema del matching come richiesto.

\subsection{Vertex Coloring}

\begin{figure}
\fig{0.9}{MIS-coloring-reduction}
\caption{Esempio di figura}
\label{Figura:Esempio di figura}
\end{figure}

\begin{figure}
\fig{0.9}{MIS-coloring-reduction-with-colors}
\caption{Esempio di figura}
\label{Figura:Esempio di figura}
\end{figure}

\begin{figure}
\fig{0.9}{final-MIS-coloing-reduction}
\caption{Esempio di figura}
\label{Figura:Esempio di figura}
\end{figure}

\end{document} 
