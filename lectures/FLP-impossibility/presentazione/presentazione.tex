\documentclass{beamer}
\usepackage[italian]{babel}
\usepackage[utf8x]{inputenc}
\usepackage[T1]{fontenc}
\usepackage{eurosym}
\usepackage{tikz}
\usetheme{Berlin}
%\usetheme{PaloAlto}

\usecolortheme{dolphin}
\usetikzlibrary{snakes}

\begin{document}


\title[$\qquad\qquad\qquad\qquad\qquad\qquad\qquad\qquad\qquad\qquad\qquad\qquad\qquad\qquad\qquad\qquad\qquad\qquad\qquad$Pierucci, Nocentini, Bruni]{\Huge\textbf{Teorema FLP}}
\subtitle{\scriptsize Impossibilità del consenso distribuito in presenza di un processore guasto}
\author[Teorema FLP - Impossibilità del consenso distribuito in presenza di un processore guasto]{Chiara Pierucci, Massimo Nocentini, Valentino Bruni}
\date{18 dicembre 2013}
\frame{\titlepage}

\begin{frame}{Indice}\tableofcontents\end{frame}


\section{Introduzione}
\begin{frame}\begin{block}{}\centering\LARGE Introduzione\end{block}\vspace{0.5cm}\end{frame}


\begin{frame}\frametitle{Consenso}
Processori collaborano per raggiungere una decisione comune 
\begin{itemize}
\item no decisione banale \begin{itemize}\item valore concordato con scambio di messaggi\end{itemize}
\item no decisione esterna \begin{itemize}\item valore proposto dai processori\end{itemize}
\end{itemize}
\vspace*{.5cm}
Ambiente asincrono uniforme
\\\vspace*{.5cm}
Possibilità di guasti di tipo ``stop''
\end{frame}


\begin{frame}\frametitle{Impossibilità del consenso}
\begin{block}{Teorema FLP (Fischer, Lynch, Paterson, 1985)}Non esistono protocolli di consenso totalmente corretti, nonostante la presenza di un guasto.\end{block}
\end{frame}

\begin{frame}\frametitle{Scambio di messaggi}
Schema asincrono
\begin{itemize}
\item\emph{consegna}: se un messaggio $m$ è stato inviato ad un processore
  $p$, prima o poi $p$ riceverà $m$
\item\emph{ritardo}: un processore $p$ non può fare assunzioni su quando riceverà i messaggi a lui destinati
\item\emph{ordine}: i messaggi destinati ad un processore $p$ possono non
  essere consegnati nell'ordine in cui sono stati inviati
\end{itemize}
\end{frame}


\begin{frame}\frametitle{Formalizzazione}
Protocollo di consenso
\begin{itemize}
\item sistema asincrono composto da $n\geq 2$ processori
\item stato del processore $p$
\begin{itemize}
\item memoria interna infinita
\item registro di input binario $x_p$
\item registro di output $y_p$ con valori in $\{0,1,b\}$
\begin{itemize}\item stato di decisione se $y_p\in\{0,1\}$\end{itemize}
\end{itemize}
\end{itemize}

Messaggio
\begin{itemize}
\item coppia $(p,m)$, dove $p$ destinatario e $m$ contenuto
\item evento $e=(p,m)$ quando $p$ riceve la coppia $(p,m)$
\end{itemize}
\end{frame}




\begin{frame}\frametitle{Message buffer}
Struttura dati astratta
\begin{itemize}
\item contenente messaggi inviati ma non ancora ricevuti
\item che supporta operazioni di 
\begin{itemize}
\item\texttt{send$(p,m)$}: il messaggio $m$ è inserito nel buffer ed è pronto per essere recapitato al processore $p$
\item\texttt{receive$(p)$}: viene recapitato a $p$
\begin{itemize}
\item un messaggio $(p,m)$ presente nel buffer
\item oppure il messaggio speciale $\perp$
\end{itemize}
\end{itemize}
\end{itemize}
\end{frame}


\begin{frame}\frametitle{Grafo di esecuzione}
\begin{figure}\centering \begin{tikzpicture}[scale=1.5]
\tikzset{nodo/.style={circle,minimum size=20pt,inner sep=0pt,draw, top color=white ,bottom color=cyan!20, cyan,text=black}}
\tikzset{nodo1/.style={circle,minimum size=20pt,inner sep=0pt, color=black}}

  \draw (0,0) node[nodo](p) {$C_{\alpha}$};
  \draw (2,0) node[nodo] (q){$C_{\beta}$}; 
  \draw(1,0.35) node[nodo1](e){\scriptsize{$e=(p,m)$}};
  
  \draw [->, line width=1pt] (p) to [bend left=15] (q);
    
 \end{tikzpicture}
 \end{figure}\vspace*{.25cm}
Grafo $G=(V,E)$ dove
\begin{itemize}
\item $V$ insieme di configurazioni
\item $E$ insieme di step
\end{itemize}
\vspace*{.25cm}
Cammino nel grafo
\begin{itemize}
\item run $\sigma$, sequenza finita o infinita di eventi
\end{itemize}
\vspace*{.25cm}
$G$ rappresenta tutte le possibili esecuzioni del protocollo
\end{frame}

\begin{frame}\frametitle{Configurazione}
Una configurazione è una tupla $C=\left<MB, (p_1,\ldots,p_n)\right>$\\\vspace*{.5cm}
Configurazione iniziale
\begin{itemize}
\item $MB=\emptyset$
\item $y_{p_i}=b$, $\forall i\in\{1,\ldots,n\}$
\end{itemize}
\vspace*{.25cm}
Configurazione di decisione
\begin{itemize}
\item $\exists i\in\{1,\ldots,n\}$: $y_{p_i}\in\{0,1\}$
\end{itemize}
\begin{figure}\centering\begin{tikzpicture}[scale=0.25]

\tikzset{label/.style={minimum size=15pt,inner sep=0pt,},}
\tikzset{conf/.style={circle,minimum size=15pt,inner sep=0pt,draw, top color=white ,bottom color=cyan!20, cyan,text=blue},}

\tikzset{camminoUp/.style={->, line width=1pt, snake=coil, segment amplitude=1pt, segment aspect=0, segment length=4mm}}
\tikzset{camminoDown/.style={->,line width=1pt, snake=coil, mirror snake, segment amplitude=1pt, segment aspect=0, segment length=4mm}}
	\draw
		(0,0) node[conf] (Ci) {$C_i$}
		(10,3) node[conf] (C1) {$C_\alpha$}
		(10, -3) node[conf] (C2) {$C_\beta$}
		(20, 4.5) node[conf] (C11) {}
		(20, 1.5) node[conf] (C12) {}
		(20, -4.5) node[conf] (C21) {}
		(20, -1.5) node[conf] (C22) {};


	\draw[camminoUp](Ci) -- (C1) node [label, above=7pt] (TextNode) {$\langle bb1b \rangle$};
	\draw[camminoDown](Ci) -- (C2) node [label, above=7pt] (TextNode) {$\langle 0bbb \rangle$};
	\draw[camminoUp](C1) -- (C11) node [label, right=10pt] (TextNode) {$\langle b11b \rangle$};
	\draw[camminoDown](C1) -- (C12) node [label, right=10pt] (TextNode) {$\langle 1b1b \rangle$};
	\draw[camminoDown](C2) -- (C21)  node [label, right=10pt] (TextNode) {$\langle 00bb \rangle$};
	\draw[camminoUp](C2) -- (C22)  node [label, right=10pt] (TextNode) {$\langle 0bb0 \rangle$};

\end{tikzpicture}\end{figure}
\end{frame}

\begin{frame}\frametitle{Run}
Run ammissibile
\begin{itemize}
\item ogni configurazione $C$ contiene al più un processore guasto
\item tutti i messaggi sono stati recapitati
\end{itemize}
\vspace*{0.25cm}
Run decisionale
\begin{itemize}
\item esiste una configurazione di decisione.
\end{itemize}
\end{frame}

\begin{frame}\frametitle{Protocollo di consenso}
\begin{block}{Protocollo parzialmente corretto}
\begin{itemize}
\item nessuna configurazione raggiungibile ha più di un valore di decisione
\item esiste una configurazione raggiungibile con valore di decisione 0 e una con valore di decisione 1.
\end{itemize}
\end{block}
\pause
\begin{block}{Protocollo totalmente corretto}
\begin{itemize}
\item parzialmente corretto
\item ogni run ammissibile è decisionale
\end{itemize}
\end{block}
\end{frame}

\section{Risultato di impossibilità}
\begin{frame}{}\begin{block}{}\centering\LARGE Risultato di impossibilità\end{block}\vspace{0.5cm}\end{frame}


\begin{frame}\frametitle{Teorema FLP}
\begin{block}{Teorema FLP - Non esistono protocolli di consenso totalmente corretti, nonostante la presenza di un guasto}
Supponiamo per assurdo che esista un protocollo totalmente corretto a meno di un guasto, dimostriamo che
\begin{enumerate}
\item esiste una configurazione iniziale bivalente
\item esistono configurazioni 0-valenti e 1-valenti 
\item esiste una configurazione bivalente non iniziale
\item esiste un run ammissibile ma non decisionale
\end{enumerate}
\end{block}
\end{frame}

\begin{frame}\frametitle{1 - Esiste una configurazione iniziale bivalente}
$$C_\alpha = (x_1,x_2,\ldots,x_{i-1},\fbox{0},x_{i+1},\ldots,x_n)\quad\mbox{0-valente}$$
$$C_\beta = (x_1,x_2,\ldots,x_{i-1},\fbox{1},x_{i+1},\ldots,x_n)\quad\mbox{1-valente}$$
\centering \scriptsize guasto$\quad\ $
\begin{columns}[c]
    \column{.5\textwidth}
		\begin{figure}\centering\begin{tikzpicture}[scale=0.15]

\tikzset{label/.style={minimum size=15pt,inner sep=0pt,},}
\tikzset{conf/.style={circle,minimum size=15pt,inner sep=0pt,draw, top color=white ,bottom color=cyan!20, cyan,text=blue},}

\tikzset{camminoUp/.style={->, line width=1pt, snake=coil, segment amplitude=1pt, segment aspect=0, segment length=4mm}}
\tikzset{camminoDown/.style={->,line width=1pt, snake=coil, mirror snake, segment amplitude=1pt, segment aspect=0, segment length=4mm}}
	\draw
		(0,0) node[conf] (Ca) {$C_\alpha$}
		(20,5) node[conf] (a1) {}
		(20,0) node[conf] (a2) {}
		(20,-5) node[conf] (a3) {}
		;
	

	\draw[camminoUp](Ca) -- (a1);
	\draw[camminoUp](Ca) -- (a2) node [label, right=10pt] (TextNode) {$\langle 0 \rangle$};
	\draw(10,1) node [label] (TextNode) {$\sigma$}; 
	\draw[camminoUp](Ca) -- (a3);


	\draw
		(0,-15) node[conf] (Cb) {$C_\beta$}
		(20,-10) node[conf] (b1) {}
		(20,-15) node[conf] (b2) {}
		(20,-20) node[conf] (b3) {}
		;
	\draw[camminoUp](Cb) -- (b1) node [label, right=10pt] (TextNode) {$\langle 1 \rangle$};
	\draw[camminoUp](Cb) -- (b2);
	\draw[camminoUp](Cb) -- (b3);


\end{tikzpicture}\end{figure}
    \column{.5\textwidth}
    		 \begin{center}\color{white}{$C_\beta$ bivalente}\end{center}
\end{columns}
\end{frame}

\begin{frame}\frametitle{1 - Esiste una configurazione iniziale bivalente}
$$C_\alpha = (x_1,x_2,\ldots,x_{i-1},\fbox{0},x_{i+1},\ldots,x_n)\quad\mbox{0-valente}$$
$$C_\beta = (x_1,x_2,\ldots,x_{i-1},\fbox{1},x_{i+1},\ldots,x_n)\quad\mbox{1-valente}$$
\centering \scriptsize guasto$\quad\ $
\begin{columns}[c]
    \column{.5\textwidth}
		\begin{figure}\centering\input{img/teo1b.tex}\end{figure}
    \column{.5\textwidth}
    		\begin{center}$C_\beta$ bivalente\end{center}
\end{columns}
\end{frame}

\begin{frame}\frametitle{2 - Esistono configurazioni 0-valenti e 1-valenti}
\begin{center}\color{black}{$E_0$ configurazione con decisione 0}\end{center}
\begin{figure}\centering\begin{tikzpicture}[scale=.65]


\tikzset{label/.style={minimum size=15pt,inner sep=0pt,},}
\tikzset{conf/.style={circle,minimum size=15pt,inner sep=0pt,draw, top color=white ,bottom color=cyan!20, cyan,text=blue},}

\tikzset{camminoUp/.style={->, line width=1pt, snake=coil, segment amplitude=1pt, segment aspect=0, segment length=4mm}}
\tikzset{camminoDown/.style={->, line width=1pt, snake=coil, mirror snake, segment amplitude=1pt, segment aspect=0, segment length=4mm}}
	\draw
		(0,-1) node[conf] (c) {$C$}
		(1,-4) node[conf] (e0) {$E_0$}
		(2,-7) node[conf] (f0) {$F_0$}
		(4,-8) node[label] (x) {}	
		
		(-2,-8) node[label] (e1) {}	
;

	\draw[->,line width=1pt](e0) to [bend left=15]  (f0);\draw(2,-5.5)node[label]{$e$};
	\draw[camminoUp](f0) -- (x);
	\draw[camminoUp](c) -- (e0);


\draw (0,-4) ellipse (2.5cm and 1cm);\draw (2,-3) node (x) {$\mathcal{C}$};
\draw (1,-7) ellipse (2.5cm and 1cm);\draw (3,-6) node (x) {$\mathcal{D}$};





	
\end{tikzpicture}\end{figure}
\begin{center}\color{black}{analogamente per $E_1$ con decisione 1}\end{center}
\end{frame}

\begin{frame}\frametitle{2 - Esistono configurazioni 0-valenti e 1-valenti}
\begin{center}\color{black}{$E_0$ configurazione con decisione 0}\end{center}
\begin{figure}\centering\begin{tikzpicture}[scale=.65]


\tikzset{label/.style={minimum size=15pt,inner sep=0pt,},}
\tikzset{conf/.style={circle,minimum size=15pt,inner sep=0pt,draw, top color=white ,bottom color=cyan!20, cyan,text=blue},}

\tikzset{camminoUp/.style={->, line width=1pt, snake=coil, segment amplitude=1pt, segment aspect=0, segment length=4mm}}
\tikzset{camminoDown/.style={->, line width=1pt, snake=coil, mirror snake, segment amplitude=1pt, segment aspect=0, segment length=4mm}}
	\draw
		(0,-1) node[conf] (c) {$C$}
	
		(4,-8) node[label] (x) {}	


		(-1,-4) node[conf] (cp) {$C'$}
		(0,-7) node[conf] (f1) {$F'_0$}	
		(-2,-8) node[conf] (e1) {$E_0$}	

		;


	\draw[camminoDown](c) -- (cp);
	\draw[->,line width=1pt](cp) to [bend right=15]  (f1);\draw(-1.1,-5.5)node[label]{$e$};
	\draw[camminoDown](f1) -- (e1);

\draw (0,-4) ellipse (2.5cm and 1cm);\draw (2,-3) node (x) {$\mathcal{C}$};
\draw (1,-7) ellipse (2.5cm and 1cm);\draw (3,-6) node (x) {$\mathcal{D}$};





	
\end{tikzpicture}\end{figure}
\begin{center}\color{black}{analogamente per $E_1$ con decisione 1}\end{center}
\end{frame}


\begin{frame}\frametitle{3 - Esiste una configurazione bivalente non iniziale}
\begin{center}\color{black}{Configurazioni vicine raggiungono configurazioni discordi}\end{center}
\begin{figure}\centering\begin{tikzpicture}


\tikzset{label/.style={minimum size=15pt,inner sep=0pt,},}
\tikzset{conf/.style={circle,minimum size=15pt,inner sep=0pt,draw, top color=white ,bottom color=cyan!20, cyan,text=blue},}

\tikzset{camminoUp/.style={->, line width=1pt, snake=coil, segment amplitude=1pt, segment aspect=0, segment length=4mm}}
\tikzset{camminoDown/.style={->, line width=1pt, snake=coil, mirror snake, segment amplitude=1pt, segment aspect=0, segment length=4mm}}
	\draw
		(0,0) node[conf] (c1) {$C_\alpha$}
		(4,0) node[conf] (c2) {$C_\beta$}
		(8,0) node[conf] (c3) {$C_\gamma$}
		
		(0,-2) node[conf] (d1) {} (0,-2.5)node[label]{$dv=0$}
		(4,-2) node[conf] (d2) {} (4,-2.5)node[label]{$dv=1$}
		(8,-2) node[conf] (d3) {} (8,-2.5)node[label]{$dv=1$}
		
;


\draw[->,line width=1pt](c1) to [bend left=15]  (c2);\draw(2,0.65)node[label]{$e'=(p',m')$};
\draw[->,line width=1pt](c3) to [bend right=15]  (c2);\draw(6,0.65)node[label]{$e''=(p'',m'')$};
\draw[->,line width=1pt](c1) to [bend right=15]  (d1);\draw(-0.35,-1)node[label]{$e$};
\draw[->,line width=1pt](c2) to [bend right=15]  (d2);\draw(3.65,-1)node[label]{$e$};
\draw[->,line width=1pt](c3) to [bend right=15]  (d3);\draw(7.65,-1)node[label]{$e$};





\end{tikzpicture}\end{figure}
\end{frame}

\begin{frame}\frametitle{3 - Esiste una configurazione bivalente non iniziale}
\begin{center}\color{black}{$e=(p,m)$ ed $e'=(p',m')$}\end{center}
\begin{figure}\centering\begin{tikzpicture}[scale=0.75]


\tikzset{label/.style={minimum size=15pt,inner sep=0pt,},}
\tikzset{conf/.style={circle,minimum size=15pt,inner sep=0pt,draw, top color=white ,bottom color=cyan!20, cyan,text=blue},}

\tikzset{camminoUp/.style={->, line width=1pt, snake=coil, segment amplitude=1pt, segment aspect=0, segment length=4mm}}
\tikzset{camminoDown/.style={->, line width=1pt, snake=coil, mirror snake, segment amplitude=1pt, segment aspect=0, segment length=4mm}}



	\draw
		(6,4) node[conf] (d0) {$d_0$}
		(8,6) node[conf] (c0) {$C_\alpha$}
		(10,6) node[conf] (c1) {$C_\beta$}
		(12,0) node[label] (e1) {}
		(12,4) node[conf] (d1) {$d_1$};
		
	\draw[->,line width=1pt](c0) to [bend right=15] (d0);\draw(6.75,5.5)node[label]{$e$};

	\draw[->,line width=1pt](c0) to [bend left=15] (c1);\draw(9,6.45)node[label]{$e'$};
	\draw[->,line width=1pt](c1) to [bend left=15] (d1);\draw(11.25,5.5)node[label]{$e$};

	
\end{tikzpicture}\end{figure}
\begin{center}\color{white}{assurdo se $p=p'$}\end{center}
\end{frame}

\begin{frame}\frametitle{3 - Esiste una configurazione bivalente non iniziale}
\begin{center}\color{black}{$e=(p,m)$ ed $e'=(p',m')$}\end{center}
\begin{figure}\centering\input{img/teo3b.tex}\end{figure}
\begin{center}\color{black}{assurdo se $p\neq p'$}\end{center}
\end{frame}

\begin{frame}\frametitle{3 - Esiste una configurazione bivalente non iniziale}
\begin{center}\color{black}{$e=(p,m)$ ed $e'=(p,m')$}\end{center}
\begin{figure}\centering\begin{tikzpicture}[scale=0.75]


\tikzset{label/.style={minimum size=15pt,inner sep=0pt,},}
\tikzset{conf/.style={circle,minimum size=15pt,inner sep=0pt,draw, top color=white ,bottom color=cyan!20, cyan,text=blue},}

\tikzset{camminoUp/.style={->, line width=1pt, snake=coil, segment amplitude=1pt, segment aspect=0, segment length=4mm}}
\tikzset{camminoDown/.style={->, line width=1pt, snake=coil, mirror snake, segment amplitude=1pt, segment aspect=0, segment length=4mm}}



	\draw
		(6,4) node[conf] (d0) {$d_0$}
		(8,6) node[conf] (c0) {$C_\alpha$}
		(10,6) node[conf] (c1) {$C_\beta$}
		(12,0) node[label] (e1) {}
		(12,4) node[conf] (d1) {$d_1$};
		
	\draw[->,line width=1pt](c0) to [bend right=15] (d0);\draw(6.75,5.5)node[label]{$e$};

	\draw[->,line width=1pt](c0) to [bend left=15] (c1);\draw(9,6.45)node[label]{$e'$};
	\draw[->,line width=1pt](c1) to [bend left=15] (d1);\draw(11.25,5.5)node[label]{$e$};

	
\end{tikzpicture}\end{figure}
\begin{center}\color{white}{assurdo se $p=p'$}\end{center}
\end{frame}

\begin{frame}\frametitle{3 - Esiste una configurazione bivalente non iniziale}
\begin{center}\color{black}{$e=(p,m)$ ed $e'=(p,m')$}\end{center}
\begin{figure}\centering\input{img/teo3c.tex}\end{figure}
\begin{center}\color{white}{assurdo se $p=p'$}\end{center}
\end{frame}

\begin{frame}\frametitle{3 - Esiste una configurazione bivalente non iniziale}
\begin{center}\color{black}{$e=(p,m)$ ed $e'=(p,m')$}\end{center}
\begin{figure}\centering\begin{tikzpicture}[scale=0.75]


\tikzset{label/.style={minimum size=15pt,inner sep=0pt,},}
\tikzset{conf/.style={circle,minimum size=15pt,inner sep=0pt,draw, top color=white ,bottom color=cyan!20, cyan,text=blue},}

\tikzset{camminoUp/.style={->, line width=1pt, snake=coil, segment amplitude=1pt, segment aspect=0, segment length=4mm}}
\tikzset{camminoDown/.style={->, line width=1pt, snake=coil, mirror snake, segment amplitude=1pt, segment aspect=0, segment length=4mm}}



	\draw
		(6,0) node[conf] (e0) {$E_0$}
		(6,4) node[conf] (d0) {$d_0$}
		(8,2) node[conf] (A) {$A$}
		(8,6) node[conf] (c0) {$C_\alpha$}
		(10,6) node[conf] (c1) {$C_\beta$}
		(12,0) node[conf] (e1) {$E_1$}
		(12,4) node[conf] (d1) {$d_1$};
		
	\draw[->,line width=1pt](c0) to [bend right=15] (d0);\draw(6.75,5.5)node[label]{$e$};

	\draw[->,line width=1pt](c0) to [bend left=15] (c1);\draw(9,6.45)node[label]{$e'$};
	\draw[->,line width=1pt](c1) to [bend left=15] (d1);\draw(11.25,5.5)node[label]{$e$};
	\draw[camminoUp](d0) -- (e0) node [label, midway, left=1pt] (TextNode) {$\sigma$};
	\draw[camminoUp](c0) -- (A) node [label, midway, right=1pt] (TextNode) {$\sigma$};
	\draw[camminoUp](d1) -- (e1) node [label, midway, right=1pt] (TextNode) {$\sigma$};

	\draw[->,line width=1pt](A) to [bend right=15] (e0);\draw(6.8,1.5)node[label]{$e$};



\end{tikzpicture}\end{figure}
\begin{center}\color{white}{assurdo se $p=p'$}\end{center}
\end{frame}


\begin{frame}\frametitle{3 - Esiste una configurazione bivalente non iniziale}
\begin{center}\color{black}{$e=(p,m)$ ed $e'=(p,m')$}\end{center}
\begin{figure}\centering\input{img/teo3e.tex}\end{figure}
\begin{center}\color{black}{assurdo per $\sigma$ decisionale}\end{center}
\end{frame}

\begin{frame}\frametitle{4 - Esiste un run ammissibile ma non decisionale}
\begin{figure}\centering\begin{tikzpicture}[scale=.9]

\tikzset{label/.style={minimum size=15pt,inner sep=0pt,},}
\tikzset{conf/.style={circle,minimum size=25pt,inner sep=0pt,draw, top color=white ,bottom color=cyan!20, cyan,text=blue}}
\tikzset{camminoUp/.style={->,line width=1pt, snake=coil, segment amplitude=1pt, segment aspect=0, segment length=4mm}}
\tikzset{camminoDown/.style={->, line width=1pt,snake=coil, mirror snake, segment amplitude=1pt, segment aspect=0, segment length=4mm}}

	\draw
		(0,0) node[conf] (Ci) {$C_i$}
	
	
		(9, -7) node[label] (x) {}
		
		;


	
\end{tikzpicture}\end{figure}
\begin{center}\color{white}{$C_\beta$ bivalente}\end{center}
\end{frame}


\begin{frame}\frametitle{4 - Esiste un run ammissibile ma non decisionale}
\begin{figure}\centering\input{img/teo4a.tex}\end{figure}
\begin{center}\color{white}{$C_\beta$ bivalente}\end{center}
\end{frame}

\begin{frame}\frametitle{4 - Esiste un run ammissibile ma non decisionale}
\begin{figure}\centering\begin{tikzpicture}[scale=.9]

\tikzset{label/.style={minimum size=15pt,inner sep=0pt,},}
\tikzset{conf/.style={circle,minimum size=25pt,inner sep=0pt,draw, top color=white ,bottom color=cyan!20, cyan,text=blue}}
\tikzset{camminoUp/.style={->,line width=1pt, snake=coil, segment amplitude=1pt, segment aspect=0, segment length=4mm}}
\tikzset{camminoDown/.style={->, line width=1pt,snake=coil, mirror snake, segment amplitude=1pt, segment aspect=0, segment length=4mm}}

	\draw
		(0,0) node[conf] (Ci) {$C_i$}
		(3,-1) node[conf] (Cj) {$C_{j-1}$}
		(3,-3) node[conf] (Dj) {$D_j$}	


		(9, -7) node[label] (x) {}
		
		;

	\draw[camminoUp](Ci) -- (Cj) node [label, midway, above,sloped] (TextNode) {$e_1^{\alpha},\ldots,e_{j-1}^{\alpha}$};
	
	\draw[->, line width=1pt](Cj)to[bend left=15]  (Dj);
	\draw(3.5,-2) node [label] {$e_j^{\alpha}$};

	
\end{tikzpicture}\end{figure}
\begin{center}\color{white}{$C_\beta$ bivalente}\end{center}
\end{frame}

\begin{frame}\frametitle{4 - Esiste un run ammissibile ma non decisionale}
\begin{figure}\centering\input{img/teo4c.tex}\end{figure}
\begin{center}\color{white}{$C_\beta$ bivalente}\end{center}
\end{frame}

\begin{frame}\frametitle{4 - Esiste un run ammissibile ma non decisionale}
\begin{figure}\centering\input{img/teo4d.tex}\end{figure}
\begin{center}\color{white}{$C_\beta$ bivalente}\end{center}
\end{frame}

\begin{frame}\frametitle{4 - Esiste un run ammissibile ma non decisionale}
\begin{figure}\centering\begin{tikzpicture}[scale=.9]

\tikzset{label/.style={minimum size=15pt,inner sep=0pt,},}
\tikzset{conf/.style={circle,minimum size=25pt,inner sep=0pt,draw, top color=white ,bottom color=cyan!20, cyan,text=blue}}
\tikzset{camminoUp/.style={->,line width=1pt, snake=coil, segment amplitude=1pt, segment aspect=0, segment length=4mm}}
\tikzset{camminoDown/.style={->, line width=1pt,snake=coil, mirror snake, segment amplitude=1pt, segment aspect=0, segment length=4mm}}

	\draw
		(0,0) node[conf] (Ci) {$C_i$}
		(3,-1) node[conf] (Cj) {$C_{j-1}$}
		(3,-3) node[conf] (Dj) {$D_j$}	

		
		(6, -4) node[conf] (Ck) {$C_{k-1}$}
		(6, -6) node[conf] (Dk) {$D_k$}	

		(9, -7) node[label] (x) {}
		
		;

	\draw[camminoUp](Ci) -- (Cj) node [label, midway, above,sloped] (TextNode) {$e_1^{\alpha},\ldots,e_{j-1}^{\alpha}$};
	\draw[camminoUp](Dj) -- (Ck) node [label, midway, above,sloped] (TextNode) {$e_1^{\beta},\ldots,e_{k-1}^{\beta}$};

	\draw[camminoUp, dashed](Dk) -- (x);


	\draw[->, line width=1pt](Cj)to[bend left=15]  (Dj);
	\draw[->, line width=1pt](Ck)to[bend left=15]  (Dk);
	\draw(3.5,-2) node [label] {$e_j^{\alpha}$};
	\draw(6.5,-5) node [label]  {$e_k^{\beta}$};
	
\end{tikzpicture}\end{figure}
\begin{center}\color{white}{$C_\beta$ bivalente}\end{center}
\end{frame}


\section{Consenso con processori inizialmente guasti}
\begin{frame}{}\begin{block}{}\centering\LARGE Consenso con processori inizialmente guasti\end{block}\vspace{0.5cm}\end{frame}

\begin{frame}\frametitle{Consenso con processori inizialmente guasti}
Assunzioni
\begin{itemize}
\item sistema non uniforme con $n$ noto
\item maggioranza dei processori non guasti $L\geq\left\lceil\dfrac{n+1}{2}\right\rceil$
\item nessun guasto durante l'esecuzione
\end{itemize}
Obiettivo
\begin{itemize}
\item raggiungere il consenso tra i processori funzionanti
\begin{itemize}
\item informando i processori funzionanti di quali siano gli altri processori funzionanti
\end{itemize}
\end{itemize}
\end{frame}



\begin{frame}\frametitle{Protocollo di consenso con $L$ processori}
Ogni processore $p$
\begin{enumerate}
	\item<1->costruisce $G$
	\begin{enumerate}[i)]
		\item<1-> effettua un broadcast della propria ID
		\item<1-> attende di ricevere esattamente $L-1$ ID
		\begin{itemize}
			\item<1-> tali processori sono chiamati \texttt{ancestors($p$)}
		\end{itemize}
		\item<1-> costruisce il grafo $G=(V,E)$ dove
		\begin{itemize}
			\item<1-> $V$ contiene tutti gli $n$ processori
			\item $(i,j)\in E$ se $j$ riceve un messaggio da $i$
		\end{itemize}
	\end{enumerate}

	\item<2-> amplia $G$
	\begin{enumerate}[i)]
		\item<2-> effettua un broadcast di $x_p$ e delle ID degli \texttt{ancestors($p$)}
		\item<2-> aggiorna $G$ con le informazioni ricevute dal broadcast
	\end{enumerate}

	\item<3-> costruisce $G^+$
	\begin{enumerate}[i)]
		\item<3-> chiude transitivamente $G$ ottenendo $G^+$
	\end{enumerate}
	
	\item<4-> determina la clique iniziale
	\begin{enumerate}[i)]
		\item<4-> i nodi nella clique raggiungono il consenso
	\end{enumerate}
\end{enumerate}
\end{frame}


\begin{frame}\frametitle{Esempio con $n=5$ ed $L=3$}
Costruzione di $G$
\begin{figure}\centering\begin{tikzpicture}[scale=.45]

\tikzset{mynode/.style={circle,minimum size=15pt,inner sep=0pt,draw, top color=white ,bottom color=blue!20, blue,text=black},}
\tikzset{mynode2/.style={circle,minimum size=15pt,inner sep=0pt,draw, top color=white ,bottom color=magenta!20, magenta,text=black},}
\tikzset{mynode3/.style={circle,minimum size=15pt,inner sep=0pt,draw, top color=white ,bottom color=orange!20, orange,text=black},}
\tikzset{mynode6/.style={circle,minimum size=15pt,inner sep=0pt,draw, top color=white ,bottom color=green!20, green,text=black},}
\tikzset{mynode7/.style={circle,fill=gray,minimum size=15pt,inner sep=0pt,},draw}
\tikzset{label/.style={minimum size=15pt,inner sep=0pt,},}


%ALPHA  
\draw (0,0) node [mynode7] () {};
\draw (0,2) node [mynode2] (alfa1) {$\alpha$};
\draw (4,1) node [mynode] (gamma1) {$\gamma$};
\draw (2,-1) node [mynode6] (delta1) {$\delta$};
\draw (2,3) node [mynode3] (beta1) {$\beta$};

\draw[->,line width=1pt] (beta1) -- (alfa1);
\draw[->,line width=1pt] (gamma1) -- (alfa1);
\draw(2,-2) node[label] (l) {Punto di vista di $\alpha$};
  

%BETA  
\draw (10,0) node [mynode7] () {};
\draw (10,2) node [mynode2] (alfa2) {$\alpha$};
\draw (14,1) node [mynode] (gamma2) {$\gamma$};
\draw (12,-1) node [mynode6] (delta2) {$\delta$};
\draw (12,3) node [mynode3] (beta2) {$\beta$};

\draw[->,line width=1pt] (alfa2) -- (beta2);
\draw[->,line width=1pt] (gamma2) -- (beta2);
\draw(12,-2.1) node[label] (l) {Punto di vista di $\beta$};
     
 
 

%GAMMA
\draw (0,-7) node [mynode7] () {};
\draw (0,-5) node [mynode2] (alfa3) {$\alpha$};
\draw (4,-6) node [mynode] (gamma3) {$\gamma$};
\draw (2,-8) node [mynode6] (delta3) {$\delta$};
\draw (2,-4) node [mynode3] (beta3) {$\beta$};

\draw[->,line width=1pt] (alfa3) -- (gamma3);
\draw[->,line width=1pt] (beta3) -- (gamma3);
\draw(2,-9.05) node[label] (l) {Punto di vista di $\gamma$};
  


%DELTA
\draw (10,-7) node [mynode7] () {};
\draw (10,-5) node [mynode2] (alfa4) {$\alpha$};
\draw (14,-6) node [mynode] (gamma4) {$\gamma$};
\draw (12,-8) node [mynode6] (delta4) {$\delta$};
\draw (12,-4) node [mynode3] (beta4) {$\beta$};

\draw[->,line width=1pt] (alfa4) -- (delta4);
\draw[->,line width=1pt] (beta4) -- (delta4);
\draw(12,-9) node[label] (l) {Punto di vista di $\delta$};
      
     
\end{tikzpicture}\end{figure}
\end{frame}

\begin{frame}\frametitle{Esempio con $n=5$ ed $L=3$}
Ampliamento di $G$
\begin{columns}[c]
    \column{.5\textwidth}
    		\begin{itemize}
			\item[]\texttt{ancestors}$(\alpha)=\{\beta,\gamma\}$
			\item[]\texttt{ancestors}$(\beta)=\{\alpha,\gamma\}$
			\item[]\texttt{ancestors}$(\gamma)=\{\alpha,\beta\}$
			\item[]\texttt{ancestors}$(\delta)=\{\alpha,\beta\}$
    		\end{itemize}
    \column{.5\textwidth}
    		\begin{figure}\centering\begin{tikzpicture}[scale=.45]

\tikzset{mynode/.style={circle,minimum size=13pt,inner sep=0pt,draw, top color=white ,bottom color=blue!20, blue,text=black},}
\tikzset{mynode2/.style={circle,minimum size=13pt,inner sep=0pt,draw, top color=white ,bottom color=magenta!20, magenta,text=black},}
\tikzset{mynode3/.style={circle,minimum size=13pt,inner sep=0pt,draw, top color=white ,bottom color=orange!20, orange,text=black},}
\tikzset{mynode6/.style={circle,minimum size=13pt,inner sep=0pt,draw, top color=white ,bottom color=green!20, green,text=black},}
\tikzset{mynode7/.style={circle,fill=gray,minimum size=13pt,inner sep=0pt,},draw}
\tikzset{label/.style={minimum size=15pt,inner sep=0pt,},}


%PENTAGONO  
\draw (0,0) node [mynode7] () {};
\draw (0,2) node [mynode2] (alfa1) {$\alpha$};
\draw (4,1) node [mynode] (gamma1) {$\gamma$};
\draw (2,-1) node [mynode6] (delta1) {$\delta$};
\draw (2,3) node [mynode3] (beta1) {$\beta$};

\draw[->,magenta,line width=1pt] (beta1) to[bend left](alfa1);
\draw[->,magenta,line width=1pt] (gamma1) -- (alfa1);
\draw[->,orange,line width=1pt] (alfa1) to[bend left](beta1);
\draw[->,orange,line width=1pt] (gamma1)to[bend right](beta1);
\draw[->,blue,line width=1pt] (alfa1) to[bend right] (gamma1);
\draw[->,blue,line width=1pt](beta1)to[bend right]  (gamma1);    
\draw[->,green,dashed,line width=1pt] (alfa1) -- (delta1);
\draw[->,green,dashed,line width=1pt](beta1) -- (delta1);     
\draw(2,-2) node[label] (l) {Ampliamento per $\alpha$, $\beta$ e $\gamma$};
  

%PENTAGONO2  
\draw (0,-7) node [mynode7] () {};
\draw (0,-5) node [mynode2] (alfa2) {$\alpha$};
\draw (4,-6) node [mynode] (gamma2) {$\gamma$};
\draw (2,-8) node [mynode6] (delta2) {$\delta$};
\draw (2,-4) node [mynode3] (beta2) {$\beta$};

\draw[->,magenta,line width=1pt] (beta2) to[bend left](alfa2);
\draw[->,magenta,line width=1pt] (gamma2) -- (alfa2);
\draw[->,orange,line width=1pt] (alfa2) to[bend left](beta2);
\draw[->,orange,line width=1pt] (gamma2)to[bend right](beta2);
\draw[->,blue,dashed,line width=1pt] (alfa2) to[bend right] (gamma2);
\draw[->,blue,dashed,line width=1pt](beta2) to[bend right] (gamma2);    
\draw[->,green,line width=1pt] (alfa2) -- (delta2);
\draw[->,green,line width=1pt](beta2) -- (delta2);     
\draw(2,-9) node[label] (l) {Ampliamento per $\delta$};
     
 
 
     
\end{tikzpicture}\end{figure}
\end{columns}
\end{frame}


\begin{frame}\frametitle{Esempio con $n=5$ ed $L=3$}
Costruzione di $G^+$\vspace*{.5cm}
\begin{columns}[c]
    \column{.5\textwidth}
    		Astrazione del grafo $G$
    		\begin{figure}\centering\input{img/astrazione.tex}\end{figure}
    \column{.5\textwidth}
    		Chiusura transitiva di $G$
    		\begin{figure}\centering\input{img/chiusura.tex}\end{figure}
\end{columns}
\end{frame}


\begin{frame}\frametitle{Esempio con $n=5$ ed $L=3$}
Clique iniziale $IC$
$$p\in IC \ \leftrightarrow\ \forall q: q\in\texttt{ancestors}(p)\rightarrow p\in\texttt{ancestors}(q)$$
\begin{figure}\centering\begin{tikzpicture}[scale=.75]

\tikzset{mynode/.style={circle,minimum size=20pt,inner sep=0pt,draw, top color=white ,bottom color=blue!20, blue,text=black},}
\tikzset{mynode2/.style={circle,minimum size=20pt,inner sep=0pt,draw, top color=white ,bottom color=magenta!20, magenta,text=black},}
\tikzset{mynode3/.style={circle,minimum size=20pt,inner sep=0pt,draw, top color=white ,bottom color=orange!20, orange,text=black},}
\tikzset{mynode6/.style={circle,minimum size=20pt,inner sep=0pt,draw, top color=white ,bottom color=green!20, green,text=black},}
\tikzset{mynode7/.style={circle,fill=gray,minimum size=20pt,inner sep=0pt,},draw}
\tikzset{label/.style={minimum size=15pt,inner sep=0pt,},}

\tikzset{clique/.style={circle,minimum size=25pt,inner sep=0pt,draw}}

%PENTAGONO  
\draw (0,0) node [mynode7] () {};
\draw (0,2) node [mynode2] (alfa1) {$\alpha$};
\draw (4,1) node [mynode] (gamma1) {$\gamma$};
\draw (2,-1) node [mynode6] (delta1) {$\delta$};
\draw (2,3) node [mynode3] (beta1) {$\beta$};

\draw (0,2)node[clique](c1){};
\draw (2,3)node[clique](c1){};
\draw (4,1)node[clique](c1){};     

\draw[->,line width=1pt] (beta1) to[bend left](alfa1);
\draw[->,line width=1pt] (gamma1) -- (alfa1);
\draw[->,line width=1pt] (alfa1) to[bend left](beta1);
\draw[->,line width=1pt] (gamma1)to[bend right](beta1);
\draw[->,line width=1pt] (alfa1) to[bend right] (gamma1);
\draw[->,line width=1pt](beta1)to[bend right]  (gamma1);    
\draw[->,line width=1pt] (alfa1) -- (delta1);
\draw[->,line width=1pt](beta1) -- (delta1);      
\draw[->,line width=1pt](gamma1) -- (delta1);      
\draw[->,line width=1pt] (alfa1) to [loop left] (alfa1);
\draw[->,line width=1pt] (beta1) to [loop above] (beta1);
\draw[->,line width=1pt] (gamma1) to [loop right] (gamma1);

\end{tikzpicture}\end{figure}
\begin{center}I nodi $\alpha$, $\beta$ e $\gamma$ appartengono alla clique iniziale\end{center}
\end{frame}


\begin{frame}\frametitle{Proprietà della clique iniziale}
\begin{itemize}
\item Esistenza della clique iniziale
	\begin{itemize}\item $G^+$ contiene almeno una clique iniziale\end{itemize}
\item Estensione della clique iniziale
	\begin{itemize}\item Ogni clique iniziale ha almeno $L$ processori\end{itemize}
\item Unicità della clique iniziale
	\begin{itemize}\item $G^+$ contiene non più di una clique iniziale\end{itemize}
\item Conoscenza della stessa clique iniziale
	\begin{itemize}\item Ogni processore $p$ costruisce la stessa clique iniziale.\end{itemize}
\end{itemize}
\end{frame}


\begin{frame}\frametitle{$G^+$ contiene almeno una clique iniziale}
Per assurdo $\forall p: p\not\in IC$
\begin{itemize}
\item sia $q \in \texttt{ancestors}(p)$
\begin{itemize}
\item $p \not \in \texttt{ancestors}(q)$
\end{itemize}
\item analogamente sia $r \in \texttt{ancestors}(q)$ 
\begin{itemize}
\item quindi $q \not  \in \texttt{ancestors}(r)$
\end{itemize} 
\item $\ldots$
\end{itemize}
\begin{figure}\centering\begin{tikzpicture}[scale=.85]
\tikzset{nodo/.style={circle,minimum size=20pt,inner sep=0pt,draw, top color=white ,bottom color=magenta!20, magenta,text=black}}


  \draw (0,0) node[nodo](p) {$p$};
  \draw (2,.2) node[nodo] (q){$q$};  
  \draw (4,.4) node[nodo](r) {$r$};
  \draw (6,.6) node[nodo](x) {}; 
  \draw (10,1) node[nodo](y) {};
  \draw (12,1.2) node[nodo](k) {$z$};              

	\draw[->, line width=1pt] (k) to [bend right=15](y);
	\draw[->, gray!50, line width=1pt] (y)to[bend right=15] (k);   
    
    \draw[->, line width=1pt] (x) to [bend right=15](r);
	\draw[->, gray!50,line width=1pt] (r)to[bend right=15] (x);   
	
	\draw[->, line width=1pt] (r) to [bend right=15](q);
	\draw[->, gray!50,line width=1pt] (q)to[bend right=15] (r);   
	
	\draw[->, line width=1pt] (q) to [bend right=15](p);
	\draw[->, gray!50,line width=1pt] (p)to[bend right=15] (q);   
	
	\draw[dashed=5](7,0.7)--(9,0.9);
\end{tikzpicture}
\end{figure}
I processori sono in numero finito
\begin{itemize}
\item prima o poi si creerà un ciclo
\end{itemize}
\end{frame}


\begin{frame}\frametitle{Ogni clique iniziale ha almeno $L$ processori}
Supponiamo $\forall q \in \texttt{ancestors}(c)\rightarrow q\in IC$
\begin{itemize}
\item $|IC|\leq L-1$
\item $c$ ha ricevuto $L-1$ messaggi
\begin{itemize}
\item quindi $c\in\texttt{ancestors}(c)$
\end{itemize}
\item assurdo
\begin{itemize}
\item  nessun processore invia messaggi a se stesso
\end{itemize}
\end{itemize}
\begin{figure}\centering\begin{tikzpicture}[scale=.7]

\tikzset{nodo/.style={circle,minimum size=20pt,inner sep=0pt,draw, top color=white ,bottom color=magenta!20, magenta,text=black}}
\tikzset{label/.style={minimum size=15pt,inner sep=0pt,},}


  \draw (0,0) node[nodo](q1) {$q_1$};
  \draw (2,0) node[nodo](cc) {$c$};
  \draw (4,0) node[nodo](ql) {$q_{L-1}$};
  \draw (2,2) node[nodo](c) {$c$};
\draw[->,line width=1pt] (q1) to[bend left] (c);
\draw[->,line width=1pt] (ql) to[bend right] (c);
\draw[->,line width=1pt] (cc) -- (c);
\draw [thick,decoration={brace,mirror,raise=0.5cm},decorate] (0,0) -- (4,0) ;
\draw (2,-1.25) node[label](c) {$L-1$};

\draw (5,1) node[label](l) {$\equiv$};

  \draw (6,0) node[nodo](q11) {$q_1$};
  \draw (8,0) node[nodo](ql1) {$q_{L-1}$};
  \draw (7,2) node[nodo](c1) {$c$};
\draw[->,line width=1pt] (q11) to[bend left] (c1);
\draw[->,line width=1pt] (ql1) to[bend right] (c1);
\draw[->,line width=1pt] (c1) to [loop above] (c1);
\draw [thick,decoration={brace,mirror,raise=0.5cm},decorate] (6,0) -- (8,0) ;
\draw (7,-1.25) node[label](c) {$L-2$};


\end{tikzpicture}
\end{figure}
\end{frame}


\begin{frame}\frametitle{Ogni clique iniziale ha almeno $L$ processori}
Supponiamo $\exists q\in\texttt{ancestors}(c)$ tale che $q \not\in IC$,
\begin{itemize}
\item $\exists r\in\texttt{ancestors}(q)$ tale che $q\not\in\texttt{ancestors}(r)$
\item per transitività di $G^+$
\begin{itemize}
\item $r \in \texttt{ancestors} (c)$
\end{itemize}
\item per ipotesi $c \in IC$
\begin{itemize}
\item $r \in\texttt{ancestors} (c) \rightarrow c \in \texttt{ancestors}(r)$
\end{itemize}
\item assurdo
\begin{itemize}
\item per transitività $q \in\texttt{ancestors}(r)$
\end{itemize}
\end{itemize}	
\begin{figure}\centering\begin{tikzpicture}

\tikzset{nodo/.style={circle,minimum size=20pt,inner sep=0pt,draw, top color=white ,bottom color=magenta!20, magenta,text=black}}
\tikzset{label/.style={minimum size=15pt,inner sep=0pt,},}





\draw (2,-3.5) node[nodo](c2) {$c$};
\draw (4,-3.5) node[nodo](q2) {$q$};
\draw (6,-3.5) node[nodo](r2) {$r$};
\draw[->,line width=1pt] (c2) to[bend right=15] (q2);
\draw[->,line width=1pt] (q2) to[bend right=15] (c2);
\draw[->,line width=1pt] (r2) to[bend right=15] (q2);
\draw[->,line width=1pt, gray!50] (q2) to[bend right=15] (r2);
\draw[->,line width=1pt] (c2) to[bend right=30] (r2);
\draw[->,line width=1pt] (r2) to[bend right=30] (c2);


\end{tikzpicture}
\end{figure}
\end{frame}


\begin{frame}\frametitle{$G^+$ contiene non più di una clique iniziale}
Supponiamo $IC_1$ e $IC_2$ clique iniziali
\begin{itemize}
\item entrambe hanno almeno $L>n/2$ processori
\begin{itemize}
\item esiste almeno un processore $p$ in comune
\begin{itemize}
\item $p$ è raggiungibile da tutti i processori di $IC_1$ e di $IC_2$
\item tutti i processori delle due clique sono raggiungibili da $p$
\end{itemize}
\end{itemize}
\item le due clique risultano fuse a formarne una unica
\end{itemize}
\begin{figure}\centering
\begin{tikzpicture}

\newcommand\irregularcircle[2]{% radius, irregularity
  \pgfextra {\pgfmathsetmacro\len{(#1)+rand*(#2)}}
  +(0:\len pt)
  \foreach \a in {10,20,...,350}{
    \pgfextra {\pgfmathsetmacro\len{(#1)+rand*(#2)}}
    -- +(\a:\len pt)
  } -- cycle
}


\tikzset{nodo/.style={circle,minimum size=20pt,inner sep=0pt,draw, top color=white ,bottom color=magenta!20, magenta,text=black}}


  \draw (0,0) node[nodo](a) {};
  \draw (2,0) node[nodo] (p){$p$};  
  \draw (4,0) node[nodo](b) {};

	\draw[->, line width=1pt] (a) to [bend left=15](p);
	\draw[->, line width=1pt] (p) to [bend left=15](b);
	\draw[->, line width=1pt] (b) to [bend left=15](p);
	\draw[->, line width=1pt] (p) to [bend left=15](a);
	
	 \draw[rounded corners=1mm] (.5,0) \irregularcircle{2.2cm}{1mm};
	 \draw[rounded corners=1mm] (3.5,0) \irregularcircle{2.2cm}{1mm};
  
  \draw (4,2.5) node (x) {$IC_2$};
  \draw (0,2.5) node (y) {$IC_1$};


\end{tikzpicture}
\end{figure}
\end{frame}

\begin{frame}\frametitle{Ogni processore $p$ costruisce la stessa clique iniziale}
Supponiamo che $p$ e $q$ costruiscano $IC_p\neq IC_q$
\begin{itemize}
\item dal punto di vista di $q$
\begin{itemize}
\item $z\not\in IC_q$
\item $\exists \gamma \in \texttt{ancestors}(z)$ tale che $z \not \in \texttt{ancestors}(\gamma)$
\end{itemize}
\item dal punto di vista di $p$
\begin{itemize}
\item $z\in IC_p$
\item $z \in \texttt{ancestors}(\gamma)$
\end{itemize}
\end{itemize}
\begin{itemize}
\item se nella prima fase $\gamma\in\texttt{ancestors}(q)$ 
\begin{itemize}
\item $z \in\texttt{ancestors}(\gamma)$
\end{itemize}
\item altrimenti
\begin{itemize}
\item dal broadcast  $z\in\texttt{ancestors}(\gamma)$.
\end{itemize}
\end{itemize}
\begin{figure}\centering\begin{tikzpicture}[scale=.5]

\tikzset{mynode2/.style={circle,minimum size=15pt,inner sep=0pt,draw, top color=white ,bottom color=magenta!20, magenta,text=black},}
\tikzset{mynode7/.style={circle,fill=gray,minimum size=15pt,inner sep=0pt,},draw}


  \draw (-1,3) node[mynode2](gamma) {$\gamma$};
  \draw (5,3) node[mynode2](n2) {};
  \draw (2, 3) node[mynode2](z) {$z$}; 
  \draw (2,6) node[mynode2](q) {$q$};
              
              
  %edges for z
    \draw[->, line width=1pt] (n2)to[bend left=15] (z);
	\draw[->, line width=1pt] (gamma)to[bend left=15] (z);    
    
    \draw[->,line width=1pt] (z)to[bend left=15] (n2);
    
     \draw[->,line width=1pt] (z) --(q);
    \draw[->,gray!50,line width=1pt] (z)to[bend left=15] (gamma); 
     \draw[->, dash pattern=on 10pt off 3pt on 2pt off 3pt , line width=1pt] (gamma)to[bend left=15] (q);   
    
\end{tikzpicture}
\end{figure}
\end{frame}


\begin{frame}\frametitle{Correttezza}
Il protocollo è corretto
\begin{itemize}
\item in $G^+$ esiste un'unica clique iniziale
\item ogni processore costruisce la stessa clique iniziale
\item tale clique ha almeno $L$ processori
\end{itemize}
Ogni processore conosce il valore iniziale di $L$ dei processori
\begin{itemize}
\item i processori raggiungono il consenso
\end{itemize}
\end{frame}

\section*{}
\begin{frame}\frametitle{Bibliografia}
\begin{itemize}
\item Fischer M. J., Lynch N. A., Paterson M. S., \emph{Impossibility of Distributed Consensus with One Faulty Process}, in "Journal of the Association for Computing Machinery", Vol. 32, No. 2, aprie 1985, pp. 374-382
\item[]
\item Locher T., Y. A. Pignolet, R. Wattenhofer, \textit{Principles of Distribuited Computing}, Zurich, Swiss Federal Institute of Technology, 2013
\end{itemize}
\end{frame}

\begin{frame}
\begin{center}
\huge\textbf{Teorema FLP}\\\vspace*{0.5cm}
\scriptsize Impossibilità del consenso distribuito in presenza di un processore guasto\\\vspace*{1cm}
\normalsize Chiara Pierucci, Massimo Nocentini, Valentino Bruni\\\vspace*{.75cm}
18 dicembre 2013
\end{center}
\end{frame}

\end{document}