\documentclass{article}
\usepackage[utf8x]{inputenc} % codifica scrittura
\usepackage[nochapters]{classicthesis} % nochapters
\usepackage[italian]{babel}
%\usepackage{inputenc}
\usepackage[T1]{fontenc} 
\usepackage[square,numbers]{natbib} 
\usepackage{amsmath, amsthm, amssymb, tikz}
\usetikzlibrary{snakes}
\newtheorem{definizione}{Definizione}  
\newtheorem{lemma}{Lemma}
\newtheorem{teorema}{Teorema}  
\newtheorem{proprieta}{Proprietà}  
\usepackage{titletoc}

\titlecontents{section}[3em]{}{\contentslabel{1em}}{}{\titlerule*[1.5pc]{.}\contentspage}
\titlecontents{subsection}[6em]{}{\contentslabel{2em}}{}{\titlerule*[1.5pc]{.}\contentspage}

\begin{document}

\title{\rmfamily\normalfont\spacedallcaps{Teorema FLP}}
\author{\spacedlowsmallcaps{Chiara Pierucci}\\
	\spacedlowsmallcaps{Massimo Nocentini}\\
	\spacedlowsmallcaps{Valentino Bruni}}
\date{} % no date

\maketitle


\begin{abstract}
  In questo lavoro ci proponiamo di studiare il risultato di
  impossibilità del consenso in un sistema totalmente asincrono,
  dimostrato da Michael J. Fischer, Nancy A. Lynch e Michael
  S. Paterson nel 1985.

  Nella nostra esposizione intendiamo riportare i principali risultati
  raggiunti dagli autori, integrando alcuni dettagli con brevi
  interventi che abbiamo sviluppato durante lo studio.

  Questo nostro elaborato è diviso in due macro sezioni: la prima ha
  un taglio più teorico e tratta la prova d'impossibilità; la
  seconda, invece, ha un taglio più implementativo e propone uno
  schema per risolvere il problema del consenso, aggiungendo alcune
  ipotesi.
\end{abstract}
       
\tableofcontents

\section{Introduzione}
\label{sec:intro}
Prima di mostrare il risultato principale, cerchiamo di descrivere
l'ambiente in cui andiamo a lavorare, identificando i concetti
fondamentali che saranno oggetto di discussione.\\\\
Immaginiamo che un insieme finito di processori debba collaborare con
lo scopo di raggiungere una decisione comune. Per raggiungere questo
obiettivo hanno la possibilità di scambiarsi un numero finito di
messaggi ed eseguire una computazione locale (di cui non ci interessa
limitare la complessità). Lo scambio di messaggi segue uno schema
\emph{asincrono} di cui riportiamo, per chiarezza, le proprietà:
\begin{description}
\item[consegna] se un messaggio $m$ è stato inviato ad un processore
  $p$, prima o poi $p$ riceverà $m$ una e una sola volta;
\item[ritardo] un processore $p$ non può fare assunzioni sugli
  istanti di tempo in cui riceverà i messaggi a lui destinati;
\item[ordine] i messaggi destinati ad un processore $p$ possono non
  essere consegnati nell'ordine in cui sono stati inviati;
\end{description}

Nei sistemi che tratteremo si contempla la possibilità che alcuni
processori possano subire un guasto di natura ``stop'', cessando di
scambiare messaggi e di eseguire una computazione locale, escludendo
invece guasti di tipo bizantino in cui il processore continua ad
interagire ma in maniera scorretta.

Inoltre, dalle proprietà sopra elencate si evince che non è possibile rilevare
eventuali guasti in quanto un processore $p$ non è in grado di
discriminare un processore guasto da un processore che semplicemente
ha bisogno di molto tempo per eseguire la propria computazione. Di
conseguenza, nessuno di essi può escludere a priori gli altri per
prendere la propria decisione.

\section{Risultato di impossibilit\`a FLP}

Le idee introdotte nei paragrafi precedenti possono essere
formalizzate con le seguenti definizioni, nelle quali siamo
volutamente concisi e le abbiamo scritte immaginando i processori come
semplici macchine di Turing:

\begin{definizione}[Protocollo di consenso]
  Un protocollo di consenso $\mathtt{P}$ è un sistema asincrono composto da $n$
  processori, con $n\geq2$. Ogni processore $p$ possiede un registro
  di input binario $x_p$, un registro di output $y_p$ con valori in
  $\{b,0,1\}$ e una memoria interna che si suppone infinita.
\end{definizione}

\begin{definizione}[Stato interno di un processore]
  Lo stato interno di un processore è costituito dal valore dei
  registri di input, di output e dal contenuto della memoria.  Diremo
  che un processore $p$ si trova in uno stato di decisione se $y_p \in
  \lbrace 0,1 \rbrace$. 
\end{definizione}

Sottolineiamo che una volta raggiunto uno stato di decisione, un
processore $p$ non può più modificare il valore $y_p$.

\subsection{Astrazione del sistema asincrono}

In questa sezione sviluppiamo le astrazioni necessarie per catturare i
concetti e le proprietà del sistema asincrono che abbiamo descritto
in Sezione \ref{sec:intro}.

\begin{definizione}[Messaggio]
  Un messaggio è una coppia $(p,m)$ dove $p$ è il processore
  destinatario e $m$ è il contenuto da recapitare a $p$.
\end{definizione}

Per formalizzare il ritardo arbitrario e la ricezione non
necessariamente in ordine dei messaggi, introduciamo l'entità
\emph{message buffer}, unico meccanismo per lo scambio di messaggi. Il
lettore deve dunque immaginare che due processori non possono
comunicare ``direttamente'', ma devono depositare i loro messaggi
usando questa entità.

\begin{definizione}[Message buffer]
  Un \emph{message buffer} è una struttura dati simile ad un insieme,
  contenente messaggi inviati ma non ancora ricevuti dai vari
  processori. Per garantire lo scambio di messaggi, sono supportate le
  seguenti operazioni:
\begin{itemize}
\item \texttt{send$(p,m)$}: il messaggio $m$ è inserito nel buffer ed
  è pronto per essere recapitato al processore $p$;
\item \texttt{receive$(p)$}: un messaggio viene recapitato a $p$. Più
  precisamente, il message buffer decide se estrarre un messaggio
  $(p,m)$ e recapitare $m$ a $p$ oppure inviare a $p$ un messaggio speciale,
  $\perp$, lasciando inalterato il suo contenuto.
\end{itemize}
\end{definizione}
La precedente definizione mette in risalto il non determinismo dello
scambio di messaggi. Infatti, se un processore $p$ invoca la primitiva
\texttt{receive(p)}, il message buffer può arbitrariamente decidere di
inviare al processore $p$ il contenuto speciale $\perp$ anche in
presenza di un messaggio $m$. La risposta $\perp$ permette di
rappresentare il ritardo arbitrario nella ricezione di un
messaggio. Per formalizzare la proprietà di consegna di ogni
messaggio ad un processore $p$, il message buffer, a seguito di infinite
\texttt{receive(p)}, può rispondere $\perp$ un numero \emph{finito} di
volte, non conosciuto a priori. Quindi, se un processore $p$ invoca la
primitiva \texttt{receive(p)} infinite volte, ha la garanzia di
riceve, prima o poi, \emph{tutti} i messaggi a lui destinati. Infine
il message buffer non garantisce di recapitare i messaggi nell'ordine
in cui sono stati depositati. Alla luce di queste osservazioni, diremo
che un processore $p$ si dice guasto se può effettuare solo un numero
finito di \texttt{receive(p)}.

\subsection{Grafo di esecuzione}

In questa sezione formalizziamo il concetto di esecuzione del
protocollo $\mathtt{P}$, costruendo un grafo $G_\mathtt{P}$ che ha per vertici le
\emph{configurazioni} e per archi gli \emph{step}. Diamo le
definizioni di quest'ultimi due concetti:

\begin{definizione}[Configurazione]
  Una \emph{configurazione} è costituita dall'insieme degli stati
  interni dei processori e dal contenuto del message buffer. Una
  configurazione si dice \emph{iniziale} quando il message buffer è
  vuoto e lo stato interno di ogni processore $p$ è tale che $y_p =
  b$.
\end{definizione}

\begin{definizione}[Step]
  Uno \emph{step} associa due configurazioni $C_\alpha, C_\beta$
  rispetto ad un processore $p$ e a un messaggio $m \in
  M\cup\{\perp\}$. Scriviamo $C_\alpha\xrightarrow{(p,m)} C_\beta$, se
  e solo se:
  \begin{enumerate}
  \item $p$ esegue \texttt{receive($p$)} ottenendo il messaggio $m$;
  \item in base al suo stato interno computa un nuovo stato e invia un
    insieme finito di messaggi agli altri processori, modificando
     il contenuto del message buffer.
  \end{enumerate} 
  Il comportamento deterministico del processore è completamente
  determinato dalla coppia $e=(p,m)$ chiamata evento, con lieve abuso
  di notazione. Se $C_1 \xrightarrow{e} C_2$ allora indichiamo con
  $e(C_1)=C_2$ la configurazione raggiunta applicando l'evento $e$ a
  $C_1$.
\end{definizione}

Un cammino del grafo rappresenta una possibile esecuzione del
protocollo, pertanto definiamo il concetto di \emph{run}:

\begin{definizione}[Run]
  Un \emph{run} è una sequenza $\sigma$, finita o infinita, di eventi
  che può essere applicata ad una configurazione $C$; se $\sigma$ è
  finita indichiamo con $\sigma(C)$ la configurazione raggiunta
  eseguendo $\sigma$ a partire da $C$. Una configurazione si dice
  \emph{accessibile} se è raggiungibile da una configurazione
  iniziale.
\end{definizione}

Nel seguente Lemma riportiamo un risultato generale che ci sarà
utile nelle prove successive. L'idea alla base è quella di
considerare due run in cui insiemi disgiunti di processori ricevono
messaggi: se applichiamo i due run ad una stessa configurazione,
arriveremo in due configurazioni distinte, in cui i messaggi destinati
ai processori non coinvolti nel run applicato saranno disponibili e,
quindi, possiamo scambiare l'applicazione dei run, raggiungendo la
stessa configurazione come rappresentato in Figura \ref{fig:concatenazioneSchedule}.
\begin{figure}[!h]
  \centering \begin{tikzpicture}[scale=0.5]

\tikzset{label/.style={minimum size=15pt,inner sep=0pt,},}
\tikzset{conf/.style={circle,minimum size=15pt,inner sep=0pt,draw, top color=white ,bottom color=cyan!20, cyan,text=blue}}
\tikzset{camminoUp/.style={->, snake=coil, segment amplitude=1pt, segment aspect=0, segment length=4mm}}
\tikzset{camminoDown/.style={->, snake=coil, mirror snake, segment amplitude=1pt, segment aspect=0, segment length=4mm}}

	\draw
		(0,0) node[conf] (Ci) {$C$}
		(5,2) node[conf] (C1) {$C_1$}
		(5, -2) node[conf] (C2) {$C_2$}
		(10, 0) node[conf] (Cf) {$C'$};

	\draw[camminoUp](Ci) -- (C1) node [label, midway, above left] (TextNode) {$\sigma_1$};
	\draw[camminoDown](Ci) -- (C2) node [label, midway, below left] (TextNode) {$\sigma_2$};
	\draw[camminoUp](C1) -- (Cf) node [label, midway, above right] (TextNode) {$\sigma_2$};
	\draw[camminoDown](C2) -- (Cf) node [label, midway, below right] (TextNode) {$\sigma_1$};

\end{tikzpicture}\caption{Run
    disgiunti concatenati.}\label{fig:concatenazioneSchedule}
\end{figure}

\begin{lemma}
  \label{lemma:eventsInterleaving}
  Sia $C$ una configurazione e siano $\sigma_1$, $\sigma_2$ due run
  che operano su insiemi di processori disgiunti. Allora
  $\sigma_2(\sigma_1(C))=\sigma_1(\sigma_2(C))$.
\end{lemma}

\begin{proof}
  Siano $P_1$ e $P_2$ gli insiemi dei processori coinvolti
  rispettivamente in $\sigma_1$ e $\sigma_2$. Sia $C_1=\sigma_1(C)$ e
  sia $C_2=\sigma_2(C)$. In $C_1$ solo i processori in $P_1$ avranno
  ricevuto messaggi e modificato il loro stato, così come succede per
  i processori in $P_2$. Dal punto di vista dei processori in $P_1$ le
  configurazioni $C$ e $C_2$ sono equivalenti e, per quelli in $P_2$
  sono equivalenti le configurazioni $C$ e $C_1$. Applicando
  $\sigma_2$ a $C_1$ si modifica lo stato interno dei soli processori
  in $P_2$ e, analogamente, applicando $\sigma_1$ a $C_2$ si altera lo
  stato interno dei soli processori in $P_1$. Dunque, dato il
  comportamento deterministico dei processori, vale
  $\sigma_2(C_1)=\sigma_1(C_2)=C'$.
\end{proof}


\subsection{Consenso e correttezza totale del protocollo}

Questa sezione prepara il lettore alla prova del Teorema FLP,
precisando che cosa significa raggiungere il consenso, in ambiente
asincrono, per un insieme di processori. Nonostante nella realtà sia
richiesto che tutti i processori non guasti decidano lo stesso valore,
nel nostro studio ci siamo focalizzati sulla versione ``debole'' del
problema: ci interessa dimostrare che pur considerando un solo
processore guasto, esiste un run $\sigma$ in cui \emph{nessun}
processore è in grado di raggiungere una decisione univoca.

\begin{definizione}[Valore di decisione di una configurazione]
  Una configurazione $C$ si dice che \emph{ha valore di decisione} $v
  \in \{0,1\}$ se almeno un processore $p$ si trova in uno stato di
  decisione con $y_p=v$.
\end{definizione}

\begin{definizione}[Protocollo parzialmente corretto]
  Un protocollo di consenso $\mathtt{P}$ si dice \emph{parzialmente corretto}
  se il grafo di esecuzione $G_P$ soddisfa le seguenti proprietà:
\begin{itemize}
\item nessuna configurazione accessibile ha più di un valore di
  decisione;
\item esiste una configurazione accessibile con valore di decisione
  $0$ e una con valore di decisione $1$.
\end{itemize}  
\end{definizione}

\begin{figure}[!h]
  \centering \begin{tikzpicture}[scale=0.5]

\tikzset{label/.style={minimum size=15pt,inner sep=0pt,},}
\tikzset{conf/.style={circle,minimum size=15pt,inner sep=0pt,draw, top color=white ,bottom color=cyan!20, cyan,text=blue},}

\tikzset{camminoUp/.style={->, snake=coil, segment amplitude=1pt, segment aspect=0, segment length=4mm}}
\tikzset{camminoDown/.style={->, snake=coil, mirror snake, segment amplitude=1pt, segment aspect=0, segment length=4mm}}
	\draw
		(0,0) node[conf] (Ci) {$C_i$}
		(5,2) node[conf] (C1) {}
		(5, -2) node[conf] (C2) {}
		(10, 3) node[conf] (C11) {}
		(10, 1) node[conf] (C12) {}
		(10, -1) node[conf] (C21) {}
		(10, -3) node[conf] (C22) {};


	\draw[camminoUp](Ci) -- (C1) node [label, above=7pt] (TextNode) {$\langle bb1b \rangle$};
	\draw[camminoDown](Ci) -- (C2) node [label, above=7pt] (TextNode) {$\langle 0bbb \rangle$};
	\draw[camminoUp](C1) -- (C11) node [label, right=10pt] (TextNode) {$\langle b11b \rangle$};
	\draw[camminoDown](C1) -- (C12) node [label, right=10pt] (TextNode) {$\langle 1b1b \rangle$};
	\draw[camminoUp](C2) -- (C21)  node [label, right=10pt] (TextNode) {$\langle 00bb \rangle$};
	\draw[camminoDown](C2) -- (C22)  node [label, right=10pt] (TextNode) {$\langle 0bb0 \rangle$};

\end{tikzpicture}\caption{Run
    parzialmente corretti.}\label{fig:parzialmenteCorretto}
\end{figure}

In Figura \ref{fig:parzialmenteCorretto} riportiamo nelle etichette
delle configurazioni le sequenze dei contenuti dei registri di output:
osserviamo che dalla configurazione iniziale $C_i$ si raggiungono sia
configurazioni in cui un processore ha uno stato di decisione $0$, sia
configurazioni in cui un processore ha uno stato di decisione
$1$. Inoltre, da queste configurazioni con valore di decisione, è
possibile raggiungere altre configurazioni con uguale valore di
decisione.

\begin{definizione}[Run ammissibile]
  Un run $\sigma$ si dice \emph{ammissibile} se 
  \begin{itemize}
  \item per ogni configurazione $C$ raggiunta attraverso un evento
    $e\in \sigma$, $C$ contiene al più un processore guasto;
  \item esiste una configurazione $C$ raggiunta attraverso un evento
    $e\in \sigma$ tale che per ogni processore $p$ non guasto, tutti i
    messaggi contenuti nel message buffer in $C$ destinati a $p$ sono
    stati recapitati.
  \end{itemize}
\end{definizione}

\begin{definizione}[Run decisionale]
  Un run $\sigma$ si dice \emph{decisionale} se esiste una
  configurazione $C$, raggiunta attraverso un evento $e\in \sigma$, in cui
  almeno un processore raggiunge uno stato di decisione.
\end{definizione}

\begin{definizione}[Protocollo totalmente corretto nonostante la presenza di un guasto]
  Un protocollo di consenso $\mathtt{P}$ si dice \emph{totalmente corretto},
  nonostante la presenza di un guasto, se:
\begin{itemize}
\item  $\mathtt{P}$ è parzialmente corretto;
\item per ogni run $\sigma$ ammissibile allora $\sigma$ è decisionale.
\end{itemize}
\end{definizione}

Nel seguito della trattazione, ogni volta che useremo un run $\sigma$,
si assume che sia ammissibile. Inoltre useremo l'aggettivo
``concordi'' se due configurazioni hanno stesso valore di decisione,
``discordi'' altrimenti.

\subsection{Dimostrazione dell'impossibilit\`a del consenso}

\begin{teorema}[FLP]
Non esistono protocolli di consenso totalmente corretti, nonostante la presenza di un guasto.
\end{teorema}

\begin{proof}
  L'idea alla base della prova è quella di procedere per assurdo,
  assumendo che esista un protocollo $\mathtt{P}$ totalmente corretto a meno di
  un guasto: proveremo dei risultati intermedi per arrivare ad una
  contraddizione. La prova si articola in due fasi:
  \begin{enumerate}
  \item dimostrare l'esistenza di configurazioni iniziali da cui,
    usando due run distinte, è possibile raggiungere due
    configurazioni discordi;
  \item dimostrare l'esistenza di un run infinito tale che ogni
    configurazione attraversata non ha un valore di decisione.
  \end{enumerate}
  La prima fase assicura che esistono delle esecuzioni in cui la
  decisione non è predeterminata, mentre la seconda assicura che esiste
  una esecuzione in cui ogni processore non decide nessun valore.  Nel
  seguito useremo tre proprietà per ``catalogare'' una
  configurazione:
  \begin{itemize}
  \item una configurazione $C$ si dice \emph{bivalente} se l'insieme
    dei valori di decisione delle configurazioni raggiungibili da $C$
    è $\{0,1\}$;
  \item una configurazione $C$ si dice ${i}$-\emph{valente} se
    l'insieme dei valori di decisione delle configurazioni
    raggiungibili da $C$ è $\{i\}$, con $i\in \lbrace 0, 1
    \rbrace$;
  \item una configurazione $C$ è
    \emph{monovalente} se è o $0$-valente o $1$-valente.
  \end{itemize}

  Osserviamo che l'insieme dei valori di decisione non può essere
  vuoto in quanto, dato che il protocollo $\mathtt{P}$ è supposto totalmente
  corretto per ipotesi di assurdo, segue che il grafo di esecuzione
  contiene solo run decisionali, ovvero ogni run conterrà una
  configurazione con un valore di decisione.

\begin{lemma}
\label{lemma:bivalentInitialConfigurationExists}
Il protocollo $\mathtt{P}$ possiede una configurazione iniziale bivalente.
\end{lemma}

\begin{proof}
  Per assurdo, assumiamo che non esista alcuna configurazione iniziale
  bivalente. Dunque, tutte le configurazioni iniziali del grafo di
  esecuzione di $\mathtt{P}$ devono essere o $1$-valenti o
  $0$-valenti. Costruiamo un ordinamento su di esse usando il seguente
  criterio: una configurazione $C_\alpha$ precede $C_\beta$ se e solo
  se i valori iniziali dei processori sono gli stessi a meno di
  esattamente un processore $p$ e $x_p^{(C_\alpha)} <
  x_p^{(C_\beta)}$, dove abbiamo indicato con $x_p^{(C_i)}$ il valore
  del registro $x_p$ nella configurazione $C_i$. Usando tale
  ordinamento, dimostriamo che ne esistono due consecutive tali che
  una risulta $0$-valente e l'altra $1$-valente. Per mostrare ciò
  supponiamo che, senza perdita di generalità, tutte le configurazioni
  ordinate siano $0$-valenti. Tuttavia tale assunzione non rispetta la
  proprietà di parziale correttezza del protocollo $\mathtt{P}$, in
  particolare deve esistere una configurazione con valore di decisione
  $1$ raggiungibile da una iniziale. Sia questa configurazione
  iniziale $C_1$ e consideriamo una configurazione $C_0$ a lei
  consecutiva (quest'ultima esiste sempre per come abbiamo definito
  l'ordinamento). Dato che $C_0$ e $C_1$ sono consecutive, esse
  differiscono per un solo valore iniziale, in corrispondenza di un
  processore $p$. Costruiamo un run decisionale $\sigma$ (non possiamo
  fare altrimenti poiché $\mathtt{P}$ è supposto totalmente corretto)
  in cui il processore $p$ non è coinvolto, ovvero in cui $p$ non
  riceve messaggi. Applicando $\sigma$ a $C_0$ e a $C_1$ si
  raggiungono due configurazioni distinte in cui lo stato di ogni
  processore $p'\neq p$, è uguale in entrambe; le due configurazioni
  differiscono solo per il contenuto di $x_p$. Ma dato che $\sigma$ è
  un run decisionale solo due valori potranno essere decisi: se sarà
  deciso $0$ allora $C_1$ è bivalente, se verrà deciso $1$ allora
  $C_0$ è bivalente; tale scenario è riportato in Figura
  \ref{fig:lemma2}. In entrambi i casi abbiamo raggiunto una
  contraddizione in quanto sia $C_0$ che $C_1$ erano supposte non
  bivalenti.  \newpage
\begin{figure}[!h]
  \centering \begin{tikzpicture}[scale=.75]


\tikzset{label/.style={minimum size=15pt,inner sep=0pt,},}
\tikzset{conf/.style={circle,minimum size=15pt,inner sep=0pt,draw, top color=white ,bottom color=cyan!20, cyan,text=blue},}

\tikzset{camminoUp/.style={->, line width=1pt, snake=coil, segment amplitude=1pt, segment aspect=0, segment length=4mm}}
\tikzset{camminoDown/.style={->, line width=1pt, snake=coil, mirror snake, segment amplitude=1pt, segment aspect=0, segment length=4mm}}
	\draw
		(0,2) node[conf] (c0) {$C_0$}
		(0,0) node[conf] (c1) {$C_1$}


		(8,-1) node[conf] (c11) {}
		(4,1) node[conf] (a) {}
		
		(8,3) node[conf] (c00) {}

		(8,3.5) node[label](l0){$dv=0$}
		(8,-0.5) node[label](l1){$dv=1$}		
		(4,1.5) node[label](lv0){$dv=v$};


	\draw[camminoUp](c1) -- (c11);
	\draw[camminoDown](c1) -- (a) node [label, midway, above] (TextNode) {$\sigma$};
	\draw[camminoUp](c0) -- (c00);
	\draw[camminoDown](c0) -- (a) node [label, midway, above] (TextNode) {$\sigma$};

		


	
\end{tikzpicture}\caption{Configurazione iniziale bivalente.}\label{fig:lemma2}
\end{figure}
\end{proof}
 
\begin{lemma}
\label{lemma:oneStepFromConfigInCYieldConfigInD}
Sia $C$ una configurazione bivalente del protocollo $\mathtt{P}$, e sia
$e=(p,m)$ un evento applicabile a $C$. Sia $\mathcal{C}$ l'insieme
delle configurazioni raggiungibili da $C$ attraverso dei run $\sigma$
tali che l'evento $e$ non appare in $\sigma$, e sia $\mathcal{D}$
l'insieme di tutte le configurazioni raggiungibili applicando $e$ ad
ogni configurazione presente in $\mathcal{C}$. Allora $\mathcal{D}$
contiene una configurazione bivalente.
\end{lemma}
Osserviamo che il processore $p$ può ricevere tale messaggio in $C$ e
dunque, per definizione di $\mathcal{C}$, $p$ potrà ricevere $m$ anche
in ogni configurazione in $\mathcal{C}$.
\begin{proof}
  Per assurdo assumiamo che $\mathcal{D}$ non contenga alcuna
  configurazione bivalente, dunque ogni configurazione in
  $\mathcal{D}$ è o $0$-valente o $1$-valente. Siano $E_0$ ed $E_1$ le
  configurazioni rispettivamente $0$-valente e $1$-valente
  raggiungibili da $C$. Queste configurazioni possono essere raggiunte
  utilizzando o meno l'evento $e$. Senza perdita di generalità, ci
  focalizzeremo sullo studio di $E_0$ dal momento che lo stesso
  ragionamento si può ripetere per $E_1$; analizziamo i seguenti due
  casi riportati in Figura \ref{fig:lemma3}:


\begin{itemize}
\item $E_0 \in \mathcal{C}$ e dunque è possibile applicare l'evento
  $e$ alla configurazione $E_0$ ottenendo una configurazione $F_0 \in
  \mathcal{D}$. Infatti, applicare $e$ equivale a far ricevere il
  messaggio $m$ al processore $p$ e a raggiungere la configurazione
  $F_0 \in \mathcal{D}$. Per l'ipotesi di assurdo su $\mathcal{D}$,
  $F_0$ deve essere monovalente: non può essere $1$-valente in quanto
  contraddirebbe il valore di decisione di $E_0$ e quindi anch'essa
  risulta $0$-valente;
\item $E_0 \not \in \mathcal{C}$, dunque deve esistere una
  configurazione $C'$ raggiungibile da $C$ dalla quale, applicando
  $e$, otteniamo una nuova configurazione $F_0' \in \mathcal{D}$ dalla
  quale è possibile raggiungere $E_0$. Dato che $F_0' \in \mathcal{D}$
  allora $F_0'$ è monovalente, ma dato che raggiungerà $E_0$, che è
  $0$-valente, anche $F_0'$ deve esserlo per non violare la validità
  del protocollo.
\end{itemize}


\begin{figure}[h]
  \centering \begin{tikzpicture}[scale=.75]


\tikzset{label/.style={minimum size=15pt,inner sep=0pt,},}
\tikzset{conf/.style={circle,minimum size=15pt,inner sep=0pt,draw, top color=white ,bottom color=cyan!20, cyan,text=blue},}

\tikzset{camminoUp/.style={->, line width=1pt, snake=coil, segment amplitude=1pt, segment aspect=0, segment length=4mm}}
\tikzset{camminoDown/.style={->, line width=1pt, snake=coil, mirror snake, segment amplitude=1pt, segment aspect=0, segment length=4mm}}
	\draw
		(0,-1) node[conf] (c) {$C$}
		(1,-3) node[conf] (e0) {$E_0$}
		(2,-5) node[conf] (f0) {$F_0$};

	\draw[->,line width=1pt](e0) to [bend left=15]  (f0);\draw(1.5,-4)node[label]{$e$};
	\draw[camminoUp](c) -- (e0);

\draw (1,-3) ellipse (1.5cm and 0.5cm);\draw (-0.5,-2.5) node (x) {$\mathcal{C}$};
\draw (2,-5) ellipse (1.5cm and 0.5cm);\draw (0.5,-4.5) node (x) {$\mathcal{D}$};
\draw (1,-7) node[label] (l1) {Figura 4.a: $E_0\in\mathcal{C}$.};





\draw
		(6,0) node[conf] (c) {$C$}
		(7,-2) node[conf] (f0) {$C'$}
		(8,-4) node[conf] (f1) {$F'_0$}
		(9,-6) node[conf] (e0) {$E_0$};
		
	\draw[->,line width=1pt](f0) to [bend left=15]  (f1);\draw(7.5,-3)node[label]{$e$};
	\draw[camminoUp](c) -- (f0);
	\draw[camminoUp](f1) -- (e0);
	

\draw (7,-2) ellipse (1.5cm and 0.5cm);\draw (5.5,-1.5) node (x) {$\mathcal{C}$};
\draw (8,-4) ellipse (1.5cm and 0.5cm);\draw (6.5,-3.5) node (x) {$\mathcal{D}$};
\draw (7.5,-7) node[label] (l1) {Figura 4.b: $E_0\not\in\mathcal{C}$.};

	
\end{tikzpicture}\caption{Configurazioni bivalenti
    in $\mathcal{D}$.}\label{fig:lemma3}
\end{figure}
\newpage

Notiamo quindi che a prescindere dai due casi, $\mathcal{D}$ conterrà
una configurazione $0$-valente. Tuttavia, ripetendo un ragionamento
analogo per la configurazione $E_1$ si ottiene che $\mathcal{D}$ deve
contenere anche una configurazione $1$-valente. 

Consideriamo adesso una configurazione $C_i$ e definiamo un
ordinamento $\sqsubset$ tra coppie di run $\sigma_{j}$ e $ \sigma_{k
}$, applicabili a $C_i$, tale che:
\begin{displaymath}
  \begin{split}
    \sigma_{j} \sqsubset \sigma_{k} \leftrightarrow \exists \sigma:
    \sigma_{j} &= (\sigma, (p_{r}, m_{r}), (p_{s}, m_{s}), (p,m),
    \ldots) \wedge \\ \sigma_{k}&= (\sigma, (p_{s}, m_{s}), (p_{r},
    m_{r}), (p,m), \ldots)
  \end{split}
\end{displaymath}
e siano $\Sigma_j, \Sigma_k$  le rispettive esecuzioni del
protocollo definite rispettivamente:
\begin{displaymath}
  \begin{split}
    \Sigma_{\sigma_{j}} &= C_i \xrightarrow{\sigma} C_{\sigma}
    \xrightarrow{(p_{r}, m_{r})}C_1 \xrightarrow{(p_{s},m_{s})}
    C_{\alpha} \xrightarrow{(p,m)} D_{\alpha} \xrightarrow{\ldots}\\
    \Sigma_{\sigma_{k}} &= C_i \xrightarrow{\sigma} C_{\sigma}
    \xrightarrow{(p_{s}, m_{s})}C_2 \xrightarrow{(p_{r},m_{r})}
    C_{\beta} \xrightarrow{(p,m)} D_{\beta} \xrightarrow{\ldots}\\
  \end{split}
\end{displaymath}
Per il Lemma \ref{lemma:bivalentInitialConfigurationExists},
consideriamo una configurazione iniziale bivalente $C_i$ ed ordiniamo
i possibili run a partire da $C_i$ usando $\sqsubset$, in cui la
consegna del messaggio $(p,m)$ viene ritardata. Dato che $C_i$ \`e
bivalente esistono due esecuzioni $\Sigma_{\sigma_{\alpha}},
\Sigma_{\sigma_{\beta}}$, con $\sigma_{\alpha} \sqsubset
\sigma_{\beta}$, che attraversano due configurazioni in cui,
rispettivamente, viene deciso il valore $0$ ed il valore $1$. Studiamo
adesso le due configurazioni $C_{\alpha}, C_{\beta}$ in cui il
messaggio $(p,m)$ viene consegnato, ovvero $e=(p,m)$ permette a
$C_\alpha$ di raggiungere $d_0$ e a $C_\beta$ di raggiungere $d_1$ e,
senza perdita di generalità, supponiamo che $d_0$ sia 0-valente e che
$d_1$ sia 1-valente. Procediamo per casi ed analizziamo un altro
possibile evento $e' = (p', m')$ applicabile, senza perdita di
generalit\`a, a $C_{\alpha}$, come rappresentato in Figura
\ref{fig:vicine}:



% Due configurazioni si dicono \emph{vicine} se una è raggiungibile
% dall'altra con l'applicazione di esattamente uno step, ovvero
% permettendo ad un processore $p$ di ricevere un messaggio $m$.

% \begin{lemma}
%   \label{lemma:neighborsConfigsYieldDiscordingConfigs}
%   Esiste una coppia $C_{\alpha}, C_{\beta} \in \mathcal{C}$ di
%   configurazioni vicine tale che, detta $D_\alpha=e(C_\alpha)$ e detta
%   $D_\beta=e(C_\beta)$, senza perdita di generalità, risultano
%   rispettivamente 0-valente e 1-valente.
% \end{lemma}
% \begin{proof}
%   Supponiamo, per assurdo, che ogni coppia di configurazioni vicine
%   debba raggiungere lo stesso valore di decisione. Data l'esistenza di
%   una configurazione iniziale bivalente, esisteranno due coppie di
%   configurazioni vicine $(C_{\alpha}, C_{\beta})$ e $(C_{\gamma},
%   C_{\eta})$ tali che $C_{\alpha}$ e $C_{\beta}$ sono, senza perdere
%   di generalità, $0$-valenti mentre $C_{\gamma}$ e $C_{\eta}$
%   $1$-valenti. Ma dato che i run necessari per raggiungere ognuna di
%   esse sono run ammissibili, supponiamo che in $C_{\alpha}$ esista un
%   processore $q$ capace di invocare \texttt{receive(q)} infinite
%   volte: nel momento in cui $q$ ha ricevuto tutti i messaggi a lui
%   destinati, il message buffer restituirà il messaggio $(q, \perp)$
%   permettendo da $C_{\alpha}$ di raggiungere ogni altra
%   configurazione, e quindi anche $C_{\gamma}$. Ma questo non è
%   possibile in quanto $C_{\alpha}$ e $C_{\gamma}$ sarebbero vicine
%   dalle quali si raggiungono configurazioni discordi. In Figura
%   \ref{fig:induzione} è rappresentato l'assurdo appena dimostrato.
% \end{proof}


% \begin{figure}[!h]
%   \centering \begin{tikzpicture}


\tikzset{label/.style={minimum size=15pt,inner sep=0pt,},}
\tikzset{conf/.style={circle,minimum size=15pt,inner sep=0pt,draw, top color=white ,bottom color=cyan!20, cyan,text=blue},}

\tikzset{camminoUp/.style={->, line width=1pt, snake=coil, segment amplitude=1pt, segment aspect=0, segment length=4mm}}
\tikzset{camminoDown/.style={->, line width=1pt, snake=coil, mirror snake, segment amplitude=1pt, segment aspect=0, segment length=4mm}}
	\draw
		(0,0) node[conf] (d1) {}(0,-.5)node[label](){$dv=0$}
		(2,0) node[conf] (d2) {}(2,-.5)node[label](){$dv=0$}
		(5,0) node[conf] (d3) {}(5,-.5)node[label](){$dv=1$}
		(7,0) node[conf] (d4) {}(7,-.5)node[label](){$dv=1$}
		
		(0,2) node[conf] (c1) {$C_\alpha$}
		(2,2) node[conf] (c2) {$C_\beta$}
		(5,2) node[conf] (c3) {$C_\gamma$}
		(7,2) node[conf] (c4) {$C_\delta$}

		(3.5,5) node[conf] (c) {$C$}
;

\draw[camminoUp](c) -- (c1);
\draw[camminoUp](c) -- (c2);
\draw[camminoUp](c) -- (c3);
\draw[camminoUp](c) -- (c4);
\draw[camminoUp](c1) -- (d1);
\draw[camminoUp](c2) -- (d2);
\draw[camminoUp](c3) -- (d3);
\draw[camminoUp](c4) -- (d4);

\draw[->,line width=1pt](c1) to [bend left=15]  (c2);
\draw[->,line width=1pt](c3) to [bend left=15]  (c4);
\draw[->,line width=1pt,gray!50](c1) to [bend left=30]  (c3);



\end{tikzpicture}\caption{Configurazioni vicine discordi.}\label{fig:induzione}
% \end{figure}



\begin{itemize}
\item Supponiamo che $p \neq p'$: per il Lemma
  \ref{lemma:eventsInterleaving}, possiamo applicare l'evento $e'$ a
  $d_0$ raggiungendo $d_1$, ovvero si riesce a passare da una
  configurazione $0$-valente ad una $1$-valente. Questo non è
  possibile in quanto si violerebbe la correttezza del protocollo.

\item Supponiamo che $p=p'$. Notiamo che non \`e possibile usare il
  Lemma \ref{lemma:eventsInterleaving} per dedurre $d_0
  \xrightarrow{e'} d_1$ in quanto l'insieme dei processori coinvolti
  dai due eventi $e$ e $e'$ non \`e disgiunto: in questo caso i due
  insiemi sono composti da un unico processore, $p$. Allora esiste un
  run decisionale $\sigma$, nel quale si ritarda ancora la consegna
  del messaggio $(p,m)$ (di conseguenza, $\sigma$ non coinvolge $p$) e
  sia $A$ una configurazione tale che $A=\sigma(C_{\alpha})$. Tale run
  risulta applicabile, per il Lemma \ref{lemma:eventsInterleaving},
  anche alle configurazioni $d_0'$, con $d_0' = e'(d_0)$, e $d_1$
  raggiungendo rispettivamente $E_0$ e $E_1$ che, per la correttezza
  del protocollo, devono mantenere lo stesso valore di decisione di
  $d_0$ e di $d_1$. Osserviamo che nella configurazione $A$ esistono
  almeno due messaggi pendenti per $p$ ed, in particolare, si pu\`o
  applicare sia la concatenazione $e' \circ e$, sia la concatenazione
  $e \circ e'$. Risulta che $e'(e(A))=E_0$ e che $e(e'(A))=E_1$,
  ovvero la configurazione $A$ è bivalente, contraddicendo l'ipotesi
  che $\sigma$ sia un run decisionale.
\end{itemize} 
\end{proof}
\begin{figure}[!h]
  \centering \begin{tikzpicture}[scale=0.75]


\tikzset{label/.style={minimum size=15pt,inner sep=0pt,},}
\tikzset{conf/.style={circle,minimum size=15pt,inner sep=0pt,draw, top color=white ,bottom color=cyan!20, cyan,text=blue},}

\tikzset{camminoUp/.style={->, line width=1pt, snake=coil, segment amplitude=1pt, segment aspect=0, segment length=4mm}}
\tikzset{camminoDown/.style={->, line width=1pt, snake=coil, mirror snake, segment amplitude=1pt, segment aspect=0, segment length=4mm}}
	\draw
		(0,1) node[conf] (d0) {$d_0$}
		(2,1) node[conf] (d1) {$d_1$}
		(0,5) node[conf] (c0) {$C_\alpha$}
		(2,5) node[conf] (c1) {$C_\beta$};

	\draw[->,line width=1pt](c0) to [bend left=15]  (c1);\draw(1,5.5)node[label]{$e'$};
	\draw[->,line width=1pt](d0) to [bend right=15]  (d1);\draw(1,1.2)node[label]{$e'$};

	\draw[->,line width=1pt](c0) to [bend right=15]  (d0);\draw(-0.5,3)node[label]{$e$};
	\draw[->,line width=1pt](c1) to [bend right=15]  (d1);\draw(1.5,3)node[label]{$e$};

\draw (1,1) ellipse (3cm and 1cm);\draw (-2,1.5) node (x) {$\mathcal{D}$};
\draw (1,5) ellipse (3cm and 1cm);\draw (-2,5.5) node (x) {$\mathcal{C}$};

		\draw(1,-1) node[label] (l) {Figura 6.a: $p\neq p'$.};

		\draw(9,-1) node[label] (l) {Figura 6.b: $p=p'$.};



	\draw
		(6,0) node[conf] (e0) {$E_0$}
		(6,4) node[conf] (d0) {$d_0$}
		(8,2) node[conf] (A) {$A$}
		(8,6) node[conf] (c0) {$C_\alpha$}
		(10,6) node[conf] (c1) {$C_\beta$}
		(10,2) node[conf] (x) {}
		(12,0) node[conf] (e1) {$E_1$}
		(12,4) node[conf] (d1) {$d_1$};

	\draw[->,line width=1pt](c0) to [bend right=15] (d0);\draw(6.75,5.5)node[label]{$e$};

	\draw[->,line width=1pt](c0) to [bend left=15] (c1);\draw(9,6.45)node[label]{$e'$};
	\draw[->,line width=1pt](c1) to [bend left=15] (d1);\draw(11.25,5.5)node[label]{$e$};
	
	
	\draw[->,line width=1pt](x) to [bend left=15] (e1);\draw(11.2,1.5)node[label]{$e$};

	\draw[->,line width=1pt](A) to [bend right=15] (e0);\draw(6.8,1.5)node[label]{$e$};
	\draw[camminoUp](d0) -- (e0) node [label, midway, left=1pt] (TextNode) {$\sigma$};
	\draw[camminoUp](c0) -- (A) node [label, midway, right=1pt] (TextNode) {$\sigma$};

	\draw[->,line width=1pt](A) to [bend left=15] (x);\draw(9,2.45)node[label]{$e'$};
	\draw[->,line width=1pt](x) to [bend left=15] (e1);\draw(11.2,1.5)node[label]{$e$};
	\draw[camminoUp](d1) -- (e1) node [label, midway, right=1pt] (TextNode) {$\sigma$};

	
\end{tikzpicture}\caption{$\mathcal{D}$ contiene configurazioni bivalenti.}\label{fig:vicine}
\end{figure}

Abbiamo quindi tutti gli elementi per terminare la dimostrazione
iniziale, mostrando come si riesca a costruire un run ammissibile ma
non decisionale, ovvero che attraversa infinite
configurazioni solo bivalenti.

Grazie al Lemma \ref{lemma:bivalentInitialConfigurationExists}
sappiamo che esiste una configurazione iniziale bivalente, sia questa
$C_i$. Consideriamo un processore $p_{\alpha}$ e sia
$e_1^{\alpha},\ldots,e_j^{\alpha}$ una sequenza di eventi tali che
$$C_i \xrightarrow{e_1^{\alpha}=(p_{\alpha},\perp)} C_1
\xrightarrow{e_2^{\alpha}=(p_{\alpha},\perp)} \ldots
\xrightarrow{e_{j-1}^{\alpha}=(p_{\alpha},\perp)} C_{j-1}
\xrightarrow{e_j^{\alpha}=(p_{\alpha},m)} D_{j}.$$ Si ha che le
configurazioni $C_1, \ldots, C_{j-1} \in \mathcal{C}$ rispetto
all'evento $e_j^{\alpha}=(p_{\alpha},m)$ e quindi, per il Lemma
\ref{lemma:oneStepFromConfigInCYieldConfigInD}, esiste $D_j \in
\mathcal{D}$ tale che $D_j$ è bivalente. Consideriamo adesso un nuovo
processore $p_{\beta}$ e costruiamo una sequenza di eventi
$e_1^{\beta},\ldots,e_k^{\beta}$ come fatto per il processore
$p_{\alpha}$ ed applichiamo tale sequenza alla configurazione $D_j$,
raggiungendo la configurazione $D_k$. Dato che $D_j$ è bivalente,
ancora per il Lemma \ref{lemma:oneStepFromConfigInCYieldConfigInD},
anche $D_k$ lo è. Ripetendo questo procedimento un numero arbitrario
di volte (la scelta di un processore $p_{\eta}$ è sempre possibile in
quanto esiste al più un processore guasto) usiamo le sequenze
costruite per definire un run $\sigma$ composto dalla loro
concatenazione, ovvero $\sigma =
(e_1^{\alpha},\ldots,e_j^{\alpha},e_1^{\beta},\ldots,e_k^{\beta},\ldots)$;
si veda Figura \ref{fig:serpente}. Il run $\sigma$ è ammissibile ma
non decisionale, contraddicendo l'ipotesi di aver supposto il
protocollo $\mathtt{P}$ totalmente corretto a meno di un guasto.
\end{proof}


\begin{figure}[!h]
  \centering \begin{tikzpicture}

\tikzset{label/.style={minimum size=15pt,inner sep=0pt,},}
\tikzset{conf/.style={circle,minimum size=15pt,inner sep=0pt,draw, top color=white ,bottom color=cyan!20, cyan,text=blue}}
\tikzset{camminoUp/.style={->,line width=1pt, snake=coil, segment amplitude=1pt, segment aspect=0, segment length=4mm}}
\tikzset{camminoDown/.style={->, line width=1pt,snake=coil, mirror snake, segment amplitude=1pt, segment aspect=0, segment length=4mm}}

	\draw
		(0,0) node[conf] (Ci) {$C_i$}
		(3,-1) node[conf] (Cj) {$C_{j-1}$}
		(8, -2) node[conf] (Ck) {$C_{k-1}$}
		(11.5, -2.5) node[label] (x) {}
		(5, -1) node[conf] (Dj) {$D_j$}	
		(10, -2) node[conf] (Dk) {$D_k$}	
		;

	\draw[camminoUp](Ci) -- (Cj) node [label, midway, above,sloped] (TextNode) {$e_1^{\alpha},\ldots,e_{j-1}^{\alpha}$};
	\draw[camminoDown](Dj) -- (Ck) node [label, midway, above,sloped] (TextNode) {$e_1^{\beta},\ldots,e_{k-1}^{\beta}$};

	\draw[camminoUp, dashed](Dk) -- (x);


	\draw[->, line width=1pt](Cj)to[bend left=15]  (Dj);
	\draw[->, line width=1pt](Ck)to[bend left=15]  (Dk);
	\draw(4,-0.5) node [label] {$e_j^{\alpha}$};
	\draw(9,-1.5) node [label]  {$e_k^{\beta}$};
	
\end{tikzpicture}\caption{Esistenza di un run
    ammissibile ma non decisionale.}\label{fig:serpente}
\end{figure}
\newpage
\section{Consenso in presenza di processori inizialmente guasti}

Il protocollo che ci accingiamo a presentare si occupa di risolvere il
problema del consenso supponendo di avere inizialmente la maggioranza
dei processori non guasti e nessun guasto durante l'esecuzione del
protocollo. Assumiamo che i processori non sappiano in anticipo quali
di loro sono guasti ma di lavorare in un sistema non uniforme dove il
numero $n$ dei processori presenti è noto, pertanto almeno
$L=\left\lceil\dfrac{n+1}{2}\right\rceil$ processori si assumono non
guasti. 

Il problema è dunque quello di informare i processori funzionanti di
quali siano gli altri processori funzionanti al fine di raggiungere il
consenso.

\subsection{Costruzione di $G$}
In primo luogo ogni processore effettua un broadcast di un messaggio
contenente la propria \textsc{id} e attende di ricevere esattamente
$L-1$ messaggi, per collezionare $L-1$ \textsc{id} degli altri
processori, denoteremo con $\texttt{ancestors}(p)$ questo
insieme. Ricevuti tali messaggi, ogni processore costruisce un grafo
$G=(V,E)$ in cui $V$ contiene tutti gli $n$ processori mentre $E$
contiene l'arco $(i,j)$ se $j$ riceve un messaggio da $i$.

Illustriamo con un esempio questa prima fase. Consideriamo $n=5$ con
$1$ processore guasto e rappresentiamo in Figura \ref{fig:povIniziali}
i punti di vista dei nodi funzionanti. In particolare $\alpha$ riceve
messaggi da $\beta$ e da $\gamma$, $\beta$ da $\alpha$ e da $\gamma$;
infine sia $\gamma$ che $\delta$ ricevono messaggi da $\alpha$ e da
$\beta$.

\begin{figure}[!h]
  \centering \begin{tikzpicture}[scale=.45]

\tikzset{mynode/.style={circle,minimum size=15pt,inner sep=0pt,draw, top color=white ,bottom color=blue!20, blue,text=black},}
\tikzset{mynode2/.style={circle,minimum size=15pt,inner sep=0pt,draw, top color=white ,bottom color=magenta!20, magenta,text=black},}
\tikzset{mynode3/.style={circle,minimum size=15pt,inner sep=0pt,draw, top color=white ,bottom color=orange!20, orange,text=black},}
\tikzset{mynode6/.style={circle,minimum size=15pt,inner sep=0pt,draw, top color=white ,bottom color=green!20, green,text=black},}
\tikzset{mynode7/.style={circle,fill=gray,minimum size=15pt,inner sep=0pt,},draw}
\tikzset{label/.style={minimum size=15pt,inner sep=0pt,},}


%ALPHA  
\draw (0,0) node [mynode7] () {};
\draw (0,2) node [mynode2] (alfa1) {$\alpha$};
\draw (4,1) node [mynode] (gamma1) {$\gamma$};
\draw (2,-1) node [mynode6] (delta1) {$\delta$};
\draw (2,3) node [mynode3] (beta1) {$\beta$};

\draw[->,line width=1pt] (beta1) -- (alfa1);
\draw[->,line width=1pt] (gamma1) -- (alfa1);
\draw(2,-2) node[label] (l) {Punto di vista di $\alpha$};
  

%BETA  
\draw (10,0) node [mynode7] () {};
\draw (10,2) node [mynode2] (alfa2) {$\alpha$};
\draw (14,1) node [mynode] (gamma2) {$\gamma$};
\draw (12,-1) node [mynode6] (delta2) {$\delta$};
\draw (12,3) node [mynode3] (beta2) {$\beta$};

\draw[->,line width=1pt] (alfa2) -- (beta2);
\draw[->,line width=1pt] (gamma2) -- (beta2);
\draw(12,-2.1) node[label] (l) {Punto di vista di $\beta$};
     
 
 

%GAMMA
\draw (0,-7) node [mynode7] () {};
\draw (0,-5) node [mynode2] (alfa3) {$\alpha$};
\draw (4,-6) node [mynode] (gamma3) {$\gamma$};
\draw (2,-8) node [mynode6] (delta3) {$\delta$};
\draw (2,-4) node [mynode3] (beta3) {$\beta$};

\draw[->,line width=1pt] (alfa3) -- (gamma3);
\draw[->,line width=1pt] (beta3) -- (gamma3);
\draw(2,-9.05) node[label] (l) {Punto di vista di $\gamma$};
  


%DELTA
\draw (10,-7) node [mynode7] () {};
\draw (10,-5) node [mynode2] (alfa4) {$\alpha$};
\draw (14,-6) node [mynode] (gamma4) {$\gamma$};
\draw (12,-8) node [mynode6] (delta4) {$\delta$};
\draw (12,-4) node [mynode3] (beta4) {$\beta$};

\draw[->,line width=1pt] (alfa4) -- (delta4);
\draw[->,line width=1pt] (beta4) -- (delta4);
\draw(12,-9) node[label] (l) {Punto di vista di $\delta$};
      
     
\end{tikzpicture}\caption{Situazione iniziale.}\label{fig:povIniziali}
\end{figure}


\subsection{Ampliamento di $G$}
Ogni processore $p$ effettua un broadcast diffondendo il proprio
valore iniziale e l'insieme dei processori da cui ha ricevuto un
messaggio nella fase precedente. Questo secondo broadcast fa si che
ogni processore $p$ riceva informazioni sia dagli $L-1$ processori già
``conosciuti'' nella prima fase, sia dai restanti processori non
guasti, aggiornando il proprio grafo $G$ con le nuove informazioni.



Nel nostro esempio $\alpha$ si aspetta di ricevere messaggi da $\beta$
e da $\gamma$ ma, per effetto del broadcast, riceverà un messaggio
anche da $\delta$ come rappresentato in Figura
\ref{fig:ampliamento}. La sua conoscenza dopo aver collezionato questi
messaggi comprende
$\texttt{ancestors}(\beta)\cup\texttt{ancestors}(\gamma)\cup\texttt{ancestors}(\delta)$.

\begin{figure}[!h]
  \centering \begin{tikzpicture}[scale=.45]

\tikzset{mynode/.style={circle,minimum size=13pt,inner sep=0pt,draw, top color=white ,bottom color=blue!20, blue,text=black},}
\tikzset{mynode2/.style={circle,minimum size=13pt,inner sep=0pt,draw, top color=white ,bottom color=magenta!20, magenta,text=black},}
\tikzset{mynode3/.style={circle,minimum size=13pt,inner sep=0pt,draw, top color=white ,bottom color=orange!20, orange,text=black},}
\tikzset{mynode6/.style={circle,minimum size=13pt,inner sep=0pt,draw, top color=white ,bottom color=green!20, green,text=black},}
\tikzset{mynode7/.style={circle,fill=gray,minimum size=13pt,inner sep=0pt,},draw}
\tikzset{label/.style={minimum size=15pt,inner sep=0pt,},}


%PENTAGONO  
\draw (0,0) node [mynode7] () {};
\draw (0,2) node [mynode2] (alfa1) {$\alpha$};
\draw (4,1) node [mynode] (gamma1) {$\gamma$};
\draw (2,-1) node [mynode6] (delta1) {$\delta$};
\draw (2,3) node [mynode3] (beta1) {$\beta$};

\draw[->,magenta,line width=1pt] (beta1) to[bend left](alfa1);
\draw[->,magenta,line width=1pt] (gamma1) -- (alfa1);
\draw[->,orange,line width=1pt] (alfa1) to[bend left](beta1);
\draw[->,orange,line width=1pt] (gamma1)to[bend right](beta1);
\draw[->,blue,line width=1pt] (alfa1) to[bend right] (gamma1);
\draw[->,blue,line width=1pt](beta1)to[bend right]  (gamma1);    
\draw[->,green,dashed,line width=1pt] (alfa1) -- (delta1);
\draw[->,green,dashed,line width=1pt](beta1) -- (delta1);     
\draw(2,-2) node[label] (l) {Ampliamento per $\alpha$, $\beta$ e $\gamma$};
  

%PENTAGONO2  
\draw (0,-7) node [mynode7] () {};
\draw (0,-5) node [mynode2] (alfa2) {$\alpha$};
\draw (4,-6) node [mynode] (gamma2) {$\gamma$};
\draw (2,-8) node [mynode6] (delta2) {$\delta$};
\draw (2,-4) node [mynode3] (beta2) {$\beta$};

\draw[->,magenta,line width=1pt] (beta2) to[bend left](alfa2);
\draw[->,magenta,line width=1pt] (gamma2) -- (alfa2);
\draw[->,orange,line width=1pt] (alfa2) to[bend left](beta2);
\draw[->,orange,line width=1pt] (gamma2)to[bend right](beta2);
\draw[->,blue,dashed,line width=1pt] (alfa2) to[bend right] (gamma2);
\draw[->,blue,dashed,line width=1pt](beta2) to[bend right] (gamma2);    
\draw[->,green,line width=1pt] (alfa2) -- (delta2);
\draw[->,green,line width=1pt](beta2) -- (delta2);     
\draw(2,-9) node[label] (l) {Ampliamento per $\delta$};
     
 
 
     
\end{tikzpicture}\caption{Situazione in seguito
    del broadcast.}\label{fig:ampliamento}
\end{figure}

\subsection{Chiusura transitiva}
Prima di descrivere la terza fase diamo un suggerimento su come
immaginarsi questo procedimento. Ogni processore $p$ ``disegna'' il
suo grafo usando colori diversi per rappresentare gli archi incidenti
ai processori ``conosciuti'' nella prima fase più quelli incidenti ai
processori ``conosciuti'' via broadcast nella seconda: per ogni
processore $q$ da cui $p$ ha ricevuto un messaggio, $p$ usa un colore
$c_q$, diverso dagli altri usati, per colorare gli archi incidenti a
$q$, ovvero per evidenziare l'insieme $\texttt{ancestors}(q)$.

Se adesso collezioniamo tutti i grafi disegnati e astraiamo dai
diversi colori usati dai processori, otterremo almeno $L$ copie dello
stesso grafo: possiamo quindi dedurre che almeno $L$ processori hanno
la stessa conoscenza.

Dato che tutti i processori hanno la stessa visione è possibile
chiudere transitivamente il grafo costruendo $G^+$; con questa
operazione si aggiungono eventuali self-loop e si trasformano i
cammini presenti in archi diretti.

In Figura \ref{fig:chiusura} è riportata la chiusura transitiva
dell'esempio.

\begin{figure}[!h]
  \centering \begin{tikzpicture}[scale=.75]

\tikzset{mynode/.style={circle,minimum size=20pt,inner sep=0pt,draw, top color=white ,bottom color=blue!20, blue,text=black},}
\tikzset{mynode2/.style={circle,minimum size=20pt,inner sep=0pt,draw, top color=white ,bottom color=magenta!20, magenta,text=black},}
\tikzset{mynode3/.style={circle,minimum size=20pt,inner sep=0pt,draw, top color=white ,bottom color=orange!20, orange,text=black},}
\tikzset{mynode6/.style={circle,minimum size=20pt,inner sep=0pt,draw, top color=white ,bottom color=green!20, green,text=black},}
\tikzset{mynode7/.style={circle,fill=gray,minimum size=20pt,inner sep=0pt,},draw}
\tikzset{label/.style={minimum size=15pt,inner sep=0pt,},}


%PENTAGONO  
\draw (0,0) node [mynode7] () {};
\draw (0,2) node [mynode2] (alfa1) {$\alpha$};
\draw (4,1) node [mynode] (gamma1) {$\gamma$};
\draw (2,-1) node [mynode6] (delta1) {$\delta$};
\draw (2,3) node [mynode3] (beta1) {$\beta$};

     

\draw[->,line width=1pt] (beta1) to[bend left](alfa1);
\draw[->,line width=1pt] (gamma1) -- (alfa1);
\draw[->,line width=1pt] (alfa1) to[bend left](beta1);
\draw[->,line width=1pt] (gamma1)to[bend right](beta1);
\draw[->,line width=1pt] (alfa1) to[bend right] (gamma1);
\draw[->,line width=1pt](beta1)to[bend right]  (gamma1);    
\draw[->,line width=1pt] (alfa1) -- (delta1);
\draw[->,line width=1pt](beta1) -- (delta1);      
\draw[->,line width=1pt](gamma1) -- (delta1);      
\draw[->,line width=1pt] (alfa1) to [loop left] (alfa1);
\draw[->,line width=1pt] (beta1) to [loop above] (beta1);
\draw[->,line width=1pt] (gamma1) to [loop right] (gamma1);

\end{tikzpicture}\caption{Chiusura transitiva.}\label{fig:chiusura}
\end{figure}

\subsection{Clique iniziale}
Per arrivare al consenso è necessario che $L$ processori decidano in
modo concorde; per raggiungere questo obiettivo usiamo il fatto che
hanno la stessa conoscenza per costruire una clique iniziale.
\begin{definizione}
  Considerando la chiusura transitiva $G^+$, un processore $p$
  appartiene a una clique iniziale $IC$ se e solo se $$\forall q:
  q\in\texttt{ancestors}(p)\rightarrow p\in\texttt{ancestors}(q).$$
\end{definizione} 
Ogni processore $p$ analizza ogni $q \in \texttt{ancestors(p)}$ e
determina se $q$ appartiene ad una clique iniziale.

Nel nostro esempio, i nodi $\alpha$, $\beta$ e $\gamma$ appartengono
alla clique iniziale e riescono a raggiungere il consenso.

 Dimostriamo alcune proprietà delle clique così ottenute.


\begin{teorema}[Esistenza]
\label{teorema:esistenzaInitialClique}
  $G^+$ contiene almeno una clique iniziale.
\end{teorema}
\begin{proof}
  Sia $G^+$ la chiusura transitiva di $G$. In quanto tale, $G^+$
  contiene soltanto archi diretti: se due processori $\alpha$ e
  $\beta$ sono connessi in $G$ tramite un arco, allora anche in $G^+$
  sono connessi con un arco, altrimenti se $\alpha$ e $\beta$ sono
  connessi in $G$ tramite un cammino, allora in $G^+$ lo sono tramite
  un arco. 
  
  Per definizione $$p\in IC\leftrightarrow\forall q:
  q\in\texttt{ancestors}(p)\rightarrow p\in\texttt{ancestors}(q)$$ e,
  equivalentemente, vale $$p\not\in IC\leftrightarrow\exists q:
  q\in\texttt{ancestors}(p)\wedge p\not\in\texttt{ancestors}(q).$$

  Supponiamo per assurdo che $\forall p: p\not\in IC$: scegliamo
  arbitrariamente un processore $p$ e sia $q \in \texttt{ancestors}(p)$;
  poiché nessun processore $k \in G^+$ appartiene ad una clique
  iniziale allora $p \not \in \texttt{ancestors}(q)$. Essendo anche $q
  \in G^+$ possiamo ripetere lo stesso argomento per $q$, ovvero
  esisterà un processore $r \in \texttt{ancestors}(q)$ tale che $q \not
  \in \texttt{ancestors}(r)$. \'E utile osservare che $r$ non può
  essere proprio $p$, altrimenti $p \in \texttt{ancestors}(q)$, caso
  escluso precedentemente. Possiamo continuare a scegliere in modo
  arbitrario al più $n$ processori ma, una volta arrivati all'ultimo,
  sia questo $z$, dovremmo scegliere un processore $t \in
  \texttt{ancestors}(z)$, precedentemente considerato, tale che $z
  \not \in \texttt{ancestors}(t)$. Ma scegliendo $t$ si crea un ciclo,
  condizione sufficiente per introdurre un processore nella clique,
  contraddicendo l'ipotesi che $G^+$ non contiene clique
  iniziali.
\end{proof}
\begin{figure}[!h]
  \centering \begin{tikzpicture}[scale=.85]
\tikzset{nodo/.style={circle,minimum size=20pt,inner sep=0pt,draw, top color=white ,bottom color=magenta!20, magenta,text=black}}


  \draw (0,0) node[nodo](p) {$p$};
  \draw (2,.2) node[nodo] (q){$q$};  
  \draw (4,.4) node[nodo](r) {$r$};
  \draw (6,.6) node[nodo](x) {}; 
  \draw (10,1) node[nodo](y) {};
  \draw (12,1.2) node[nodo](k) {$z$};              

	\draw[->, line width=1pt] (k) to [bend right=15](y);
	\draw[->, gray!50, line width=1pt] (y)to[bend right=15] (k);   
    
    \draw[->, line width=1pt] (x) to [bend right=15](r);
	\draw[->, gray!50,line width=1pt] (r)to[bend right=15] (x);   
	
	\draw[->, line width=1pt] (r) to [bend right=15](q);
	\draw[->, gray!50,line width=1pt] (q)to[bend right=15] (r);   
	
	\draw[->, line width=1pt] (q) to [bend right=15](p);
	\draw[->, gray!50,line width=1pt] (p)to[bend right=15] (q);   
	
	\draw[dashed=5](7,0.7)--(9,0.9);
\end{tikzpicture}
\caption{Esistenza della clique
    iniziale.}\label{fig:chiusura}
\end{figure}
\begin{teorema}[Estensione]
\label{teorema:estensioneInitialClique}
  Ogni clique iniziale ha almeno $L$ processori.
\end{teorema}
\begin{proof}
  Assumiamo per assurdo che esista una clique iniziale $IC$ tale che
  $|IC|\leq L-1$.  Consideriamo la prima fase del protocollo dove ogni
  processore non guasto riceve esattamente $L-1$ messaggi. Per il
  Teorema \ref{teorema:esistenzaInitialClique}, esiste almeno un
  processore non guasto $c \in IC$, il quale avrà ricevuto $L-1$
  messaggi avendo $L-1 = |\texttt{ancestors(c)}|$. Studiamo due casi
  in Figura \ref{fig:estensione}.
  \begin{itemize}
  \item Se $\forall q \in \texttt{ancestors}(c) \rightarrow q \in IC$
    allora, essendoci al massimo $L-1$ processori nella clique
    iniziale, $c$ deve essere uno dei processori da cui ha ricevuto un
    messaggio. Ma questo è impossibile poiché nella prima fase nessun
    processore invia messaggi a se stesso.
  \item Altrimenti deve esistere un processore $q \in
    \texttt{ancestors}(c)$, tale che $q \not \in IC$, quindi esiste un
    processore $r \in \texttt{ancestors} (q)$, secondo il quale $q \not \in
    \texttt{ancestors} (r)$. Dato che stiamo lavorando su $G^+$,
    allora per transitività vale $r \in \texttt{ancestors} (c)$ e
    quindi, dato che per ipotesi $c \in IC$, segue che $r \in
    \texttt{ancestors} (c) \rightarrow c \in \texttt{ancestors}
    (r)$. Quest'ultima implicazione ci permette di arrivare ad una
    contraddizione in quanto, per transitività, si ha $q \in
    \texttt{ancestors}(r)$, caso escluso precedentemente.
  \end{itemize}
\end{proof}
\begin{figure}[!h]
  \centering \begin{tikzpicture}

\tikzset{nodo/.style={circle,minimum size=20pt,inner sep=0pt,draw, top color=white ,bottom color=magenta!20, magenta,text=black}}
\tikzset{label/.style={minimum size=15pt,inner sep=0pt,},}


  \draw (0,0) node[nodo](q1) {$q_1$};
  \draw (2,0) node[nodo](cc) {$c$};
  \draw (4,0) node[nodo](ql) {$q_{L-1}$};
  \draw (2,2) node[nodo](c) {$c$};
\draw[->,line width=1pt] (q1) to[bend left] (c);
\draw[->,line width=1pt] (ql) to[bend right] (c);
\draw[->,line width=1pt] (cc) -- (c);
\draw [thick,decoration={brace,mirror,raise=0.5cm},decorate] (0,0) -- (4,0) ;
\draw (2,-1) node[label](c) {$L-1$};

\draw (5,1) node[label](l) {$\equiv$};

  \draw (6,0) node[nodo](q11) {$q_1$};
  \draw (8,0) node[nodo](ql1) {$q_{L-1}$};
  \draw (7,2) node[nodo](c1) {$c$};
\draw[->,line width=1pt] (q11) to[bend left] (c1);
\draw[->,line width=1pt] (ql1) to[bend right] (c1);
\draw[->,line width=1pt] (c1) to [loop above] (c1);
\draw [thick,decoration={brace,mirror,raise=0.5cm},decorate] (6,0) -- (8,0) ;
\draw (7,-1) node[label](c) {$L-2$};
\draw (4,-1.5) node[label](l) {Figura 11.a: $q_i\in IC$.};


\draw (2,-3.5) node[nodo](c2) {$c$};
\draw (4,-3.5) node[nodo](q2) {$q$};
\draw (6,-3.5) node[nodo](r2) {$r$};
\draw[->,line width=1pt] (c2) to[bend right=15] (q2);
\draw[->,line width=1pt] (q2) to[bend right=15] (c2);
\draw[->,line width=1pt] (r2) to[bend right=15] (q2);
\draw[->,line width=1pt, gray!50] (q2) to[bend right=15] (r2);
\draw[->,line width=1pt] (c2) to[bend right=30] (r2);
\draw[->,line width=1pt] (r2) to[bend right=30] (c2);
\draw (4,-5) node[label](l) {Figura 11.b: $q\not\in IC$.};


\end{tikzpicture}
\caption{Estensione della
    clique iniziale.}\label{fig:estensione}
\end{figure}

\begin{teorema}[Unicità]
  \label{teorema:unicitaInitialClique}
  $G^+$ contiene non più di una clique iniziale.\end{teorema}
\begin{proof}
  Assumiamo che esistano due clique iniziali $IC_1, IC_2$ che, per il
  Teorema \ref{teorema:estensioneInitialClique}, hanno almeno $L$
  processori e, dato che $L>n/2$, supponiamo che solo il processore
  $p$ sia in comune. Per definizione di clique iniziale, $p$ è
  raggiungibile da tutti i processori di $IC_1$ e di $IC_2$ e,
  viceversa, tutti i processori delle due clique sono raggiungibili da
  $p$. Per transitività, tutti i processori di $IC_1$ possono
  raggiungere quelli di $IC_2$ passando per $p$ così come tutti i
  processori di $IC_2$ possono raggiungere quelli di $IC_1$ passando
  per $p$. Questo scenario è rappresentato in Figura
  \ref{fig:unicita}. In tal modo le due clique risultano fuse a
  formarne una unica, contraddicendo il fatto che $IC_1$ e $IC_2$
  avevano solo $p$ in comune.
\end{proof}

\begin{figure}[!h]
  \centering 
\begin{tikzpicture}[scale=.6]

\newcommand\irregularcircle[2]{% radius, irregularity
  \pgfextra {\pgfmathsetmacro\len{(#1)+rand*(#2)}}
  +(0:\len pt)
  \foreach \a in {10,20,...,350}{
    \pgfextra {\pgfmathsetmacro\len{(#1)+rand*(#2)}}
    -- +(\a:\len pt)
  } -- cycle
}


\tikzset{nodo/.style={circle,minimum size=15pt,inner sep=0pt,draw, top color=white ,bottom color=magenta!20, magenta,text=black}}


  \draw (0,0) node[nodo](a) {};
  \draw (2,0) node[nodo] (p){$p$};  
  \draw (4,0) node[nodo](b) {};

	\draw[->, line width=1pt] (a) to [bend left=15](p);
	\draw[->, line width=1pt] (p) to [bend left=15](b);
	\draw[->, line width=1pt] (b) to [bend left=15](p);
	\draw[->, line width=1pt] (p) to [bend left=15](a);
	
	 \draw[rounded corners=1mm] (.5,0) \irregularcircle{2.2cm}{1mm};
	 \draw[rounded corners=1mm] (3.5,0) \irregularcircle{2.2cm}{1mm};
  
  \draw (4,2.5) node (x) {$IC_2$};
  \draw (0,2.5) node (y) {$IC_1$};


\end{tikzpicture}
\caption{Unicità della clique iniziale.}\label{fig:unicita}
\end{figure}

\begin{teorema}[Conoscenza]
\label{teorema:conoscenzaInitialClique}
  Ogni processore $p$, grazie a $G^+$, costruisce la stessa clique
  iniziale.
\end{teorema}
\begin{proof}
  Assumiamo per assurdo che esistano due processori $p$ e $q$ che
  costruiscono due clique iniziali $IC_p$ e $IC_q$ tali che $IC_p\neq
  IC_q$, differendo per almeno un processore $z$. Nel seguito
  assumiamo, senza perdita di generalità, che $p$ inserisca $z$ in
  $IC_p$ mentre $q$, erroneamente, non lo faccia stesso. Studiamo dal
  punto di vista di $q$: poiché $q$ ha deciso che $z$ non appartiene
  ad $IC_q$, deve esistere $\gamma \in \texttt{ancestors}(z)$ tale che
  $z \not \in \texttt{ancestors}(\gamma)$ (a torto) mentre, dal punto
  di vista di $p$, $z \in \texttt{ancestors}(\gamma)$ (a
  ragione). Distinguiamo i due casi rappresentati in Figura \ref{fig:conoscenza}:
\begin{itemize}
\item se $q$ ha ricevuto un messaggio da $\gamma$ nella prima fase,
  allora $\gamma \in \texttt{ancestors}(q)$ e nella seconda fase, $q$
  riceverà da $\gamma$ l'informazione che $z \in
  \texttt{ancestors}(\gamma)$ (in quanto questo è il caso, per la
  ragione di $p$);
\item se $q$ non ha ricevuto alcun messaggio da $\gamma$ nella prima
  fase allora riceverà un messaggio di broadcast nella seconda,
  ricavando di nuovo che $z \in \texttt{ancestors}(\gamma)$.
\end{itemize} 
In entrambi i casi $q$ ottiene la conoscenza che $z \in
\texttt{ancestors}(\gamma)$, dunque anche $q$ avrebbe dovuto inserire
$z$ in $IC_q$, contraddicendo la sua scelta.
\end{proof}

\begin{figure}[!h]
  \centering \begin{tikzpicture}

\tikzset{mynode2/.style={circle,minimum size=20pt,inner sep=0pt,draw, top color=white ,bottom color=magenta!20, magenta,text=black},}
\tikzset{mynode7/.style={circle,fill=gray,minimum size=20pt,inner sep=0pt,},draw}


  \draw (0,1) node[mynode2](gamma) {$\gamma$};
  \draw (2,0) node[mynode2] (n1){};  
  \draw (4,1) node[mynode2](n2) {};
  \draw (2, 3) node[mynode2](z) {$z$}; 
  \draw (2,5) node[mynode2](q) {$q$};
              
              
  %edges for z
    \draw[->, line width=1pt] (n1) to[bend left=15](z);
    \draw[->, line width=1pt] (n2)to[bend left=15] (z);
	\draw[->, line width=1pt] (gamma)to[bend left=15] (z);    
    
    \draw[->,line width=1pt] (z) to[bend left=15](n1);
    \draw[->,line width=1pt] (z)to[bend left=15] (n2);
    
     \draw[->,line width=1pt] (z) --(q);
    \draw[->,gray!50,line width=1pt] (z)to[bend left=15] (gamma); 
     \draw[->, dash pattern=on 10pt off 3pt on 2pt off 3pt , line width=1pt] (gamma)to[bend left=15] (q);   
    
\end{tikzpicture}
\caption{Costruzione della clique iniziale.}\label{fig:conoscenza}
\end{figure}

\begin{teorema}[Correttezza]
  Il protocollo raggiunge il consenso se almeno $L$ processori sono
  non guasti e se non si verificano ulteriori guasti durante
  l'esecuzione.\end{teorema}
\begin{proof}
  Sia $G^+$ una chiusura transitiva di un grafo relativo al protocollo
  descritto: per il Teorema \ref{teorema:esistenzaInitialClique} e per
  il Teorema \ref{teorema:unicitaInitialClique}, in $G^+$ esiste
  un'unica clique iniziale. Per il Teorema
  \ref{teorema:conoscenzaInitialClique}, ogni processore costruisce la
  stessa clique iniziale e questa ha almeno $L$ processori per il
  Teorema \ref{teorema:estensioneInitialClique}.
  
  Pertanto, dopo la
  seconda fase del protocollo, ogni processore conosce il valore
  iniziale di almeno la maggioranza dei processori ed, applicando una
  regola deterministica e stabilita a priori, è possibile
  raggiungere il consenso per gli $L$ processori non guasti.
\end{proof}



\begin{thebibliography}{}

\bibitem{FLP} Fischer M. J., Lynch N. A., Paterson M. S.,
  \emph{Impossibility of Distributed Consensus with One Faulty
    Process}, in "Journal of the Association for Computing Machinery",
  Vol. 32, No. 2, Aprile 1985, pp. 374-382.

\bibitem{ETH} Locher T., Y. A. Pignolet, R. Wattenhofer,
  \textit{Principles of Distribuited Computing}, Zurich, Swiss Federal
  Institute of Technology, 2013.


\end{thebibliography}



\end{document}
