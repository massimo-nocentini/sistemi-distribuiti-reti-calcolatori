\documentclass{article}
\usepackage{tikz}

\usepackage{tikz}
\begin{document}


\tikzset{mynode/.style={circle,minimum size=20pt,inner sep=0pt,draw, top color=white ,bottom color=blue!20, blue,text=black},}
\tikzset{mynode2/.style={circle,minimum size=20pt,inner sep=0pt,draw, top color=white ,bottom color=magenta!20, magenta,text=black},}
\tikzset{mynode3/.style={circle,minimum size=20pt,inner sep=0pt,draw, top color=white ,bottom color=orange!20, orange,text=black},}
\tikzset{mynode6/.style={circle,minimum size=20pt,inner sep=0pt,draw, top color=white ,bottom color=green!20, green,text=black},}
\tikzset{mynode7/.style={circle,fill=gray,minimum size=20pt,inner sep=0pt,},draw}

\begin{figure}
\centering
\begin{tikzpicture}
  \newdimen\R
  \R=4cm
  \draw (0:\R)
     \foreach \x in {72,144,...,360} { (\x:\R) }
              -- cycle (360:\R) node[mynode](gamma) {$\gamma$}
              -- cycle (288:\R)node[mynode6] (delta){$\delta$}
              -- cycle (216:\R) node[mynode7] {}
              -- cycle (144:\R) node[mynode2] (alfa){$\alpha$}
              -- cycle  (72:\R) node[mynode3](beta) {$\beta$};
              
              
  %edges for alpha
    \draw[->,black] (beta) -- (alfa);
    \draw[->,black] (gamma) -- (alfa);
    
     
    
     
\end{tikzpicture}
\caption{Punto di vista di $\alpha$}
\end{figure}
\end{document}