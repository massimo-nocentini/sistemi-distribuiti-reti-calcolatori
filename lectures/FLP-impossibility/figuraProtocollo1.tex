\documentclass{article}
\usepackage{tikz}
\usepackage[top=1in,bottom=1in,right=1in,left=1in]{geometry}

\begin{document} 
\tikzset{mynode/.style={circle,minimum size=20pt,inner sep=0pt,draw, top color=white ,bottom color=green!20, green,text=black},}
\tikzset{mynode2/.style={circle,fill=gray,minimum size=20pt,inner sep=0pt,},draw}
\begin{figure}
\centering
\begin{tikzpicture}
 %nodes
  \draw (-1,0) node[mynode](k) {$k$};
  \draw (0,2) node[mynode] (alfa){$\alpha$};  
  \draw (2.3,3) node[mynode](beta) {$\beta$};
  \draw (4.6, 2) node[mynode](gamma) {$\gamma$}; 
  \draw (5.6,0) node[mynode2] {};
  \draw (0,-2) node[mynode2] {};  
  \draw (2.3,-3) node[mynode] (j){$j$};
  \draw (4.6, -2) node[mynode](r) {$r$}; 
 
 %edges 
   \draw[->] (alfa) -- (k);
    \draw[->] (beta) -- (k);
    \draw[->] (gamma) -- (k);
     \draw[->] (r) -- (k); 
   
    
  \end{tikzpicture}
  \caption{Punto di vista di k}
  \end{figure}
  \end{document}